\subsection{\bf Outras}

\subparagraph{\bf Proficiência Tribal:}

Todas as 13 tribos tem uma área de competência clara, sendo assim, cada uma ganha 1 dado a mais a partir do segundo posto (Fostern) em uma das seguintes características:

\begin{enumerate}[noitemsep]
    \item Andarilhos do Asfalto: \\Segurança / Computação / Versátil*
    
    \item Crias de Fenrir: \\Armas Brancas / Intimidação / Briga
    
    \item Fianna: \\Performance / Expressão / Esquiva
    
    \item Filhos de Gaia: \\Empatia / Medicina / Iskakku
    
    \item Fúrias Negras: \\Sobrevivência / Rituais / Arqueirismo
    
    \item Garras Vermelhas: \\Instinto Primitivo / Empatia com Animais / Briga
    
    \item Peregrinos Silenciosos: \\Condução / Esportes / Armas Brancas
    
    \item Portador da Luz Interior: \\Enigmas / Linguística / Kailindô
    
    \item Presas de Prata: \\Liderança / Etiqueta / Duelo de Klaives
    
    \item Roedores de Ossos: \\Manha / Furtividade / Versátil*
    
    \item Senhores das Sombras: \\Lábia / Politica / Duelo de Klaives
    
    \item Uktena: \\Ocultismo / Investigação / Esquiva
    
    \item Wendigo: \\Prontidão / Arqueirismo / Armas Brancas

\end{enumerate}

\begin{itemize}[noitemsep]
    \item No posto 2 (Fostern) ganha +1 dado**.
    
    \item No posto 4 (Athro) ganha um sucesso automático.
    
    \item [*] Qualquer Habilidade Primária a escolha do jogador.
    
    \item [**] Este dado é considerado como um dado de especialização, não sendo contado o número 1.
\end{itemize}



\subparagraph{\bf Permanecer Ativo:}
O jogador que quiser permanecer ativo ao atingir o nível de vitalidade incapacitado deve testar sua pontuação atual ou permanente de Fúria (o que for maior) contra uma dificuldade igual a 6 para tentar vigorar o dano do ultimo ataque sofrido. Caso consiga 4 ou mais sucessos, o personagem entrará em frenesi. 6 ou mais sucessos indicam que a fúria do Garou deixou de ser pura, entrando assim no Frenesi da Wyrm, sendo preciso gastar 1 ponto de força de vontade para cada sucesso acima do 5 para resistir.

\subparagraph{\bf Regra do 1 e falhas críticas em testes de Fúria:}
Quando o valor do dado rolado é igual a '1', normalmente é anulado um sucesso para cada '1' que tenha "caído" e quando o saldo fica negativo, ou seja com mais '1s' que sucessos ocorre uma falha crítica. Porém em testes de Fúria essa não é uma regra exata, podendo variar de situação para situação a cargo do narrador.

\begin{itemize}
    \item  Em situações tensas como debates e defrontações com aliados (não inimigos).
    \begin{itemize}[noitemsep]
        \item uma falha crítica pode ser entendido como uma forma de auto controle sobre a sua fúria, você fez um teste em uma situação "crítica" e controlou sua fúria excepcionalmente bem (sendo o equivalente a um sucesso critico em testes normais).
    \end{itemize}
    
    \item  Ao se deparar com cenas grotescas ou ser confrontado por um grande inimigo (não rival)
    \begin{itemize}[noitemsep]
        \item[1] Pode ser entendida como um descontrole gigantesco sobre a sua fúria, podendo resultar em um frenesi da Wyrm.
        
        \item[2] Pode ser um medo irracional do que lhe provocou a fazer o teste de fúria, resultando num frenesi raposa.
    \end{itemize}
    
    \item  Em situações onde você busca auxilio da fúria como um teste para tentar transformar ou algum dom.
    \begin{itemize}[noitemsep]
        \item[1] Pode ser falta de afinidade com a fúria fazendo perder a possibilidade de usar ela por um tempo.
    \end{itemize}
    
    \item  Em testes para vigorar o dano ao cair abaixo de incapacitado.
    \begin{itemize}[noitemsep]
        \item Em testes para vigorar com a Fúria o '1' não anula sucesso.
    \end{itemize}
\end{itemize}

\subparagraph{\bf Frenesi:}
O personagem em frenesi recebe metade da sua pontuação permanente de Fúria por turno e deixa de sofrer as penalidades por dano e consome essa mesma quantidade de Fúria por turno para adquirir ações extras. Um Garou em frenesi pode fazer um teste de Força de Vontade com dificuldade 8 para atacar alvos específicos.
Caso a situação que tenha provocado o frenesi tenha acabado, o jogador pode testar sua Força de Vontade contra uma dificuldade igual à sua própria Fúria a cada turno para voltar a si, se o jogador não obtiver nenhum sucesso e falhar em um teste com dificuldade 10, ele permanecerá em frenesi. 

\subparagraph{\bf Dificuldade:}
A dificuldade mínima para qualquer ação em combate é 4, mesmo que possua algum outro modificador.

\subparagraph{\bf Dificuldade dos Ataques focalizados:}
Quando é declarado um ataque focalizado o jogador tem que deixar bem claro em qual parte deseja acertar, o critério da penalidade de dificuldade é baseado no tamanho da área que o jogador deseja. Acertar um tiro na cabeça é mais fácil que acertar um tiro no olho direito.
\begin{itemize}[noitemsep]
\item +1 - Qualquer área grande, o Torso, Abdômen, Costas.
\item +2 - Áreas um pouco mais especificas, Cabeça, Braço, Perna, Câmera Média.
\item +3 - Áreas muito especificas, Coração, Olho, Mão que segura a arma, Câmera Pequena.
\item +4 - Áreas extremamente especificas, Dedo Indicador, Cabo de Aço que segura a tirolesa.
\end{itemize}

\subparagraph{\bf Desmembramento:}
Ataques focalizados podem causar o desmembramento, da seguinte maneira:
\begin{itemize}[noitemsep]
\item 4 de dano, tem 50\% (Dif. 6 para resistir) de chance.
\item 5 de dano, tem 80\% (Dif. 9 para resistir) de chance.
\item 6 de dano, tem 100\% (Automático) de chance.
\end{itemize}
A porcentagem para resistir é determinada em um teste com um único dado, Dificuldade 6 (50\%) e Dificuldade 9(80\%).

\subparagraph{\bf Acerto por Dano:}
Não é difícil perceber que se tratando de Garou os danos que eles conseguem causar em combate são extremamente elevados e isso sem mencionar quando acertam com precisão seus alvos. Para não deixar a sorte tomar conta de tudo, é permitido sacrificar 3 dados de dano para diminuir a dificuldade de sucesso de infligir dano, sendo a dificuldade mínima 4.
