\documentclass[11pt]{article}

\usepackage[utf8]{inputenc}
\usepackage[brazil]{babel}
\usepackage{enumitem, multicol, graphicx, url, opensans, booktabs, float} 

%definindo tamanho da pagina
\usepackage[top=1.15cm, bottom=1.15cm, left=1.3cm, right=1.3cm]{geometry}
%package para colorir a tabela
\usepackage[table, xcdraw]{xcolor}

%apenas pra nao mudar o recipe
%\usepackage{pythontex}

%organizando borda
\usepackage[scale=1, angle=0, opacity=1]{background} 
\usepackage{tikz}
\usetikzlibrary{trees}

%\usepackage{setspace}
%\renewcommand{\baselinestretch}{1}
%\setlength{\parskip}{0.2cm}
%\setlength{\parindent}{0.2cm}

% listando sub-paragrafo e paragrafo
\setcounter{secnumdepth}{5}
\setcounter{tocdepth}{5}

\renewcommand{\thesubparagraph}{}
\renewcommand{\theparagraph}{}

\newcommand{\BackImage}[2][{scale = 1}]{
	\BgThispage
	\backgroundsetup{
      contents={
		\ifodd\value{page}
         	\includegraphics[{#1, width = \paperwidth, height = \paperheight}]{#2}
        \else
       		 \includegraphics[{#1, width = -\paperwidth, height = \paperheight}]{#2}
		\fi
        }
    }
}

\newcommand{\imgFundoTexto}[2]{
  \newpage
  \begin{tikzpicture}[remember picture, overlay]
  \tikzset{inner sep=0pt,outer sep=0pt}
  \node at (current page.center){\includegraphics[{height=\paperheight, width = \paperwidth}]{imagens/#1}};
  \end{tikzpicture}
  #2
  \newpage
}

\newcommand{\imgFundo}[1]{
  \newpage
  \begin{tikzpicture}[remember picture, overlay]
  \tikzset{inner sep=0pt,outer sep=0pt}
  \node at (current page.center){\includegraphics[{height=\paperheight, width = \paperwidth}]{imagens/#1}};
  \end{tikzpicture}
  \newpage
}

\newcommand{\imgColuna}[1]{
  \imgColunaLegenda{#1}{}
}

\newcommand{\imgColunaLegenda}[2]{
\imgColunaLegendaTexto{#1}{#2}{}
}

\newcommand{\imgColunaLegendaTexto}[3]{
\begin{center}
    \begin{tikzpicture}[remember picture]
      \tikzset{inner sep=0pt,outer sep=0pt}
      \node (at) (current page.center){
        \includegraphics[{width=0.95\linewidth}]{#1}};
    \end{tikzpicture}
    #2
\end{center}
#3
}

\newcommand{\imgColunaGraphic}[2]{
\begin{center}
    \begin{tikzpicture}[remember picture]
      \tikzset{inner sep=0pt,outer sep=0pt}
      \node (at) (current page.center){#1};
    \end{tikzpicture}
    #2
\end{center}
}

\newcommand{\preencher}{
\hfill 
\vfill
\null
}

\newcommand{\titulo}[1]{
  \begin{center}
    \section{\LARGE \bf #1}
  \end{center}
}

\newcommand{\tituloM}[4]{
  \begin{center}
    \section{\LARGE \bf #1}
    \livro{#2}, v#3 #4.
  \end{center}
}

\newcommand{\livro}[1]{Livro Base: #1}

\definecolor{vermelhoE}{rgb}{0.7, 0, 0}
\definecolor{verdeE}{rgb}{0, 0.7, 0}
\definecolor{azulE}{rgb}{0, 0, 0.7}
\definecolor{cinza}{rgb}{0.7,0.7,0.7}