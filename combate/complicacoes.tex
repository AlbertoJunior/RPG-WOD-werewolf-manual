\titulo{Complicações}

\subsection{\bf Gerais}

\subparagraph{\bf Às Cegas:}
Os personagens cegos não podem se esquivar de ataques, nem apará-los ou bloqueá-los, e recebem uma penalidade adicional igual a +2 sobre a dificuldade em todas as ações. Alguns Dons compensam a visão, mas o Narrador tem a palavra final quanto aos efeitos que esses Dons têm em combate.

\subparagraph{\bf Atordoamento:}
Como mencionado anteriormente, o personagem perderá o turno seguinte ao receber num mesmo turno uma quantidade de níveis de dano maior que seu nível de Vigor. Ele não poderá realizar ações, a não ser cambalear um pouco, e todos os ataques contra ele receberão um bônus igual a -2 sobre a dificuldade.

\subparagraph{\bf Imobilização:}
O personagem que se encontra parcialmente imobilizado e incapaz de se esquivar não numa boa posição. Todos os ataques contra esse personagem receberão um bônus igual a -2 sobre a dificuldade, se ele ainda for capaz de lutar, e serão automaticamente bem-sucedidos se ele não puder se mover de jeito nenhum.

\subparagraph{\bf Mudar a Ação:}
Normalmente, depois de ter declarado uma ação, o jogador não poderá mais mudá-la. Se tiver um bom motivo para isso (um companheiro de matilha mata o alvo visado pelo dele, por exemplo), ele poderá mudar sua ação, mas deverá acrescentar um ponto à dificuldade. Abortá-la em favor de uma ação defensiva não muda a dificuldade da dita ação.

\subparagraph{\bf Proteção:}
Alguns adversários possuem uma cobertura protetora, seja ela um casaco balístico ou uma carapaça gosmenta. A proteção acrescenta dados ao teste de absorção de um personagem. Certas proteções artificiais também restringem os movimentos, o que se reflete no aumento das dificuldades dos testes que envolvem Destreza. Alguns tipos de proteção aparecem na tabela de Proteção. Os dados de Proteção (e somente eles) podem ser usados para absorver dano letal. Esses dados também podem ser usados para absorver dano geralmente não passível de absorção, a critério do Narrador. Faria sentido um colete de Kevlar poder ser usado para absorver o dano de uma arma de prata, mas um casaco balístico não serve como proteção contra radiação.
\begin{itemize}[noitemsep]
\item Roupas Reforçadas: proteção 1, Destreza 0.
\item Colete Blindado: proteção 2, Destreza -1.
\item Colete de Kevlar: proteção 3, Destreza -1.
\item Casaco Balístico: proteção 4, Destreza -2.
\item Equipamento de Choque: proteção 5, Destreza -3.
\end{itemize}

\subparagraph{\bf Queda:}
Alguns ataques têm a intenção de derrubar o oponente. Se isto acontecer, o personagem deve voltar a ficar de pé. Se não lhe restar ações com as quais fazê-los, seu oponente poderá tratá-lo como se ele estivesse parcialmente imobilizado.