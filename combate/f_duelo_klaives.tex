\begin{center}
\section{\LARGE \bf Duelo de Klaives}
v0.1 Fabiano
\end{center}

\subparagraph{\bf Reter:} O duelista tenta prender a lâmina de seu oponente com a sua própria. Cada sucesso após o primeiro previne o oponente de executar uma ação com sua klaive naquele turno. O oponente perde qualquer Fúria que tenha gasto para ganhar ações extras. O duelista não pode fazer nada além de manter o Reter e escarnecer seu inimigo. Após um Reter bem-sucedido, o duelista pode tentar um Desarme ou um Prise d'Argent com uma dificuldade -2. Essa manobra requer a perícia específica de Duelo de Klaives. 
\begin{itemize}[noitemsep]
\item Teste: Destreza + Armas Brancas.
\item Dificuldade: Destreza + Duelos de Klaives do oponente
\item Dano: Nenhum.
\item Ações: Especial.
\end{itemize}

\subparagraph{\bf Cegar:} O Garou sangra copiosamente através de qualquer ferimento causado por uma klaive. Alguns duelistas sorrateiros tentam levar vantagem nisso cegando seus oponentes. Um simples corte na testa pode escorrer sangue nos olhos do oponente pelo resto da luta. Caso essa manobra seja bem-sucedida, a dificuldade de todos os testes de ataque do oponente, bem como os aparamentos e as esquivas, são feitos com dificuldade +1. Essa manobra requer a perícia específica de Duelo de Klaives.
\begin{itemize}[noitemsep]
\item Teste: Destreza + Duelos de Klaives
\item Dano: Arma –1.
\item Dificuldade: 8.
\item Ações: 1.
\end{itemize}

\subparagraph{\bf Desarme:} O duelista tenta girar a klaive da mão de seu oponente, num teste resistido de Força + Armas Brancas. Se o duelista for bem-sucedido, a arma de seu oponente cai a um metro dele por sucesso. Se o teste falhar, o oponente mantém sua arma. Se falhar criticamente, o duelista perde sua própria arma, que cai a uma distância em metros igual aos sucessos do seu oponente. Essa manobra requer a perícia específica de Duelo de Klaives.
\begin{itemize}[noitemsep]
\item Teste: Força + Duelos de Klaives
\item Dano: Nenhum
\item Dificuldade: 6 
\item Ações: 1
\end{itemize}

\subparagraph{\bf Finta:} O duelista finge estar atacando numa direção e então evita quaisquer tentativas de bloqueio, mirando em qualquer lugar do corpo do oponente. Esse é um teste resistido, feito contra a Percepção + Armas Brancas do oponente. O duelista pode adicionar um dado por sucesso obtido acima do total de seu oponente à sua parada de dados no seu próximo ataque. No entanto, caso seu oponente obtenha mais sucessos, o duelista perde o número de sucessos que o oponente obteve acima dele em seu próximo teste, visto que ele foi pego em sua finta e tenta recuperar a guarda. Os dados de bônus são perdidos se o próximo ataque não for feito dentro de suas ações após a finta.
\begin{itemize}[noitemsep]
\item Teste: Destreza + Duelos de Klaives
\item Dano: Nenhum
\item Dificuldade: 7 
\item Ações: 1
\end{itemize}

\subparagraph{\bf Flèche:} A manobra compreende na perda total da sutileza do duelista em um ataque total sobre seu oponente, lançando a si mesmo como uma flecha sobre ele com a lâmina como ponta. Após o golpe, o duelista estaciona próximo ao seu oponente, por trás dele, e está vulnerável por alguns segundos mortais enquanto ele recupera seu equilíbrio e se vira para encará-lo. Essa manobra requer a perícia específica de Duelo de Klaives.
\begin{itemize}[noitemsep]
\item Teste: Destreza + Esportes/Duelos de Klaives
\item Dano: Arma +4
\item Dificuldade: 7 
\item Ações: 3
\end{itemize}

\subparagraph{\bf Sondar:} Uma sonda é um golpe rápido para testar as defesas e a velocidade de reação do oponente. Uma sonda não possui a carga total da força do duelista em si, sendo apenas intencionada a medir a perícia do oponente, não para feri-lo. 
\begin{itemize}[noitemsep]
\item Teste: Destreza + Duelos de Klaives
\item Dano: Arma –2
\item Dificuldade: 5 
\item Ações: 1
\end{itemize}
    
\subparagraph{\bf Aparar:} O aparar é um movimento simples para bloquear a lâmina do oponente usando a sua própria. Se os sucessos do duelista superarem os do atacante, ele apara o golpe. Um aparar bem-sucedido permite ao duelista executar um Riposte de imediato caso ele tenha quaisquer ações restantes nesse turno, ou confere a ele um bônus de +2 na Iniciativa no turno seguinte.
\begin{itemize}[noitemsep]
\item Teste: Destreza + Duelos de Klaives
\item Dano: Nenhum
\item Dificuldade: 6 
\item Ações: 1
\end{itemize}
    
\subparagraph{\bf Prise d'Argent:} Literalmente “pressão da prata”, essa manobra, muito apreciada entre os Presas de Prata duelistas, usa a lâmina do oponente para guiar um ataque. O duelista desliza sua klaive ao longo da lâmina de seu oponente e a conduz ao seu corpo. Essa manobra apenas pode ser feita após um Riposte ou um Reter bem-sucedido. Essa manobra requer a perícia específica de Duelo de Klaives.
\begin{itemize}[noitemsep]
\item Teste: Destreza + Duelos de Klaives
\item Dano: Arma
\item Dificuldade: 4 
\item Ações: 1
\end{itemize}
    
\subparagraph{\bf Riposte:} O duelista faz um ataque rápido em seu oponente desprotegido, logo após um aparamento. Essa manobra apenas pode ser usada após um aparamento. O oponente pode tentar aparar se ele tiver quaisquer ações restantes naquele turno.
\begin{itemize}[noitemsep]
\item Teste: Destreza + Duelos de Klaives
\item Dano: Arma
\item Dificuldade: 4 
\item Ações: 1
\end{itemize}

\subparagraph{\bf Escudo de Prata:} O duelista usa sua velocidade e a envergadura da klaive para criar um “escudo de prata” em sua frente, à medida que ele gira a lâmina num padrão defensivo. Cada sucesso obtido nesse teste pode ser adicionado a quaisquer tentativas de aparar feitas durante esse turno. Essa manobra requer a perícia específica de Duelo de Klaives.
\begin{itemize}[noitemsep]
\item Teste: Destreza + Duelos de Klaives
\item Dano: Nenhum
\item Dificuldade: 7 
\item Ações: 1
\end{itemize}

\subparagraph{\bf Grande Ataque:} O duelista investe todo seu esforço em um único e massivo golpe em seu oponente, expondo a si mesmo, mas esperando causar dano suficiente para tornar isso irrelevante. As dificuldades de quaisquer seguintes ações nesse turno são aumentadas em +2. Essa manobra requer a perícia específica de Duelo de Klaives.
\begin{itemize}[noitemsep]
\item Teste: Destreza + Duelos de Klaives
\item Dano: Arma +3
\item Dificuldade: 7 
\item Ações: 2
\end{itemize}

\subparagraph{\bf Golpe de Parada:} Um duelista que possui uma vantagem na iniciativa sobre seu oponente pode escolher adiar sua ação até seu oponente agir. Caso seu oponente tente atacar, o duelista tenta invadir sua guarda e desferir um golpe rápido e mortal que para seu oponente em sua trajetória, usando sua própria força cinética para tornar o golpe ainda mais mortal. Esse ataque não pode ser aparado ou esquivado, visto que seu oponente está completamente comprometido com ao seu ataque. No entanto, se o Golpe de Parada falhar em incapacitar ou nocautear o oponente, o duelista não pode esquivar ou aparar o ataque do oponente. Essa manobra requer a perícia específica de Duelo de Klaives.
\begin{itemize}[noitemsep]
\item Teste: Destreza + Duelos de Klaives
\item Dano: Arma +2
\item Dificuldade: 7 
\item Ações: 1
\end{itemize}