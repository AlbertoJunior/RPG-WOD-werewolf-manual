\tituloM{Arquearia}{Armory}{0.1}{Alberto}

\subsection{\bf Combate com Arco \label{com:arquearia}}

\subparagraph*{\bf Pré-requisitos:} Força ••, Destreza ••, Percepção ••, Arqueirismo •••

\subsection{\bf Manobras}

\subparagraph{\bf • Braço de Arqueiro:}
Os seus braços já atiraram tantas flechas que os músculos estão bem tonificados. Seu personagem anula o pré-requisito de Força em até 1 ponto, dessa forma, com Força 2 é possível utilizar um arco composto ou Garou sem problemas.

\subparagraph{\bf •• Memória Muscular:}
Você repetiu tanto o movimento de recarregar seu arco que está ação pra você é feita de maneira automática. Seu personagem consegue recarregar o arco como se fossem ações reflexo.

\subparagraph{\bf ••• Arqueiro Olímpico:}
Você conhece a dinâmica das flechas como um mestre e sabe utilizar o vento, clima e a temperatura a seu favor, conseguindo maximar o potencial do seu arco a longas distâncias. O alcance efetivo do arco é dobrado.

\subparagraph{\bf •••• Mestre do Tiro Múltiplo:}
Utilizar o arco pra você é mais fácil do que andar, sua perícia é excepcional, não se limitando mais a atirar 1 flecha por vez, você consegue preparar até 3 flechas em um único instante, aumentando assim cadência de tiro do seu arco para 3.

\subparagraph{\bf ••••• Flecha Mergulhadora:}
Você é um mestre na arte da arquearia e provavelmente todos já sabem disso, porém você adquiriu habilidades que surpreendem seus inimigos no campo de batalha. Você consegue atirar flechas que mergulham em direção a seus alvos, ignorando a maioria das coberturas, precisando apenas de um pouco de espaço e uma visão minima de seu alvo.

\begin{itemize}[noitemsep]
\item Cobertura Simples ou Parcial: você consegue acertar qualquer parte que consiga ver sem penalidades.
\item Cobertura Média: você só precisa saber onde está o alvo e pode acerta-lo com +1 na dificuldade.
\end{itemize}