\titulo{Combate à Distância}

\subparagraph*{\bf Descrição:} Envolve qualquer arma que opere à distância, como pistolas e arcos. O personagem que usa uma arma de alcance precisa ter o alvo em seu campo de visão.

\subparagraph{\bf Alcance:}

Cada arma na tabela de armas de Alcance tem um alcance indicado. Essa distância é o alcance médio da arma: considera-se que a dificuldade é igual a 6 dento desse alcance. Uma arma pode ser disparada contra um alvo que se encontra ao dobro dessa distância, mas isso eleva a dificuldade para 8. Se o alvo estiver a menos de 1,80 metros, entretanto, a distância será considerada à queima-roupa e a dificuldade cairá para 4.

\subparagraph{\bf Apontar:}
Apontar uma área especifica (a cabeça, a mão, o peito) eleva a dificuldade em +2. Os efeitos especiais de um tiro desses ficam a cargo do Narrador.

\subparagraph{\bf Arcos:} 
Apesar da tendência dos lobisomens de considerarem as armas de fogo deselegantes e convulsivos instrumentos da Weaver, muitas tribos usam arcos em combate e transformam-nos em fetiches. Para usar um arco, o personagem precisa adquirir a Pericia Arqueirismo (uma Perícia secundária). O jogador faz um teste de Destreza + Arqueirismo para disparar uma flecha; as dificuldades para os vários tipos de arcos estão indicadas na tabela de Armas de Alcance. O jogador cujo personagem não tem a Perícia Arqueirismo pode fazer testes de Destreza + Esportes, mas esses testes estão sujeitos a uma penalidade igual a +1 sobre a dificuldade.
A flecha disparada de um arco curto a curta distância causa tanto dano quanto uma bala de pequeno calibre, e os arcos são silenciosos, de modo que seu potencial de combate é obvio. Um outro uso comum para os arcos é enfiar uma afiada seta de madeira no coração de um vampiro. Para tanto, o jogador tem de conseguir cinco sucessos a fim de acertar o coração e infligir pelo menos três níveis de dano após a absorção.
Os arcos, entretanto, têm dois problemas principais. Um deles é o fato de o ato de colocar uma flecha e puxar a corda exigir uma ação (automática), enquanto recarregar uma besta exige duas ações automáticas. O outro problema é que, se o jogador sofrer uma falha crítica no teste de ataque, a corda do arco se romperá. Se por acaso tiver uma corda sobressalente, o personagem poderá consertar o arco com um sucesso num teste de Raciocínio + Arqueirismo (ou num teste de Raciocínio + Ofícios sujeito a uma penalidade igual a +1 sobre a dificuldade). Se não a tiver, o arco não passará de uma vareta até o personagem substituir a corda.

\subparagraph{\bf Armas de Arremesso:}
Apesar de a Nação Garou não ver com bons olhos o uso de armas de fogo, as armas de arremesso são parte de quase todas as culturas. Do shuriken asiático ao chakram indiano, passando pelo bumerangue australiano - e inclusive objetos encontrados casualmente, como pedras ou veículos pequenos -, alguns Garou preferem amaciar os oponentes com ataques desse tipo antes de arremessar e entrar na batalha. O teste para usar uma arma de arremesso é Destreza + Esportes, não Armas Brancas. Geralmente, a dificuldade é igual a 6, dependendo do tamanho e distância do alvo. Se a arma utilizada não for feita para ser arremessada (a maioria das facas feitas para uso em combate próximo não são balanceadas para o arremesso, e vice-versa), o Narrador deve aumentar a dificuldade em pelo menos um ponto. Os níveis de dano para essas armas podem ser encontrados na tabela de Armas de Arremesso (Livro base 3ª edição, pag. 208). 
A distância a que uma arma pode ser arremessada com precisão e com força suficiente para provocar dano depende do peso da arma e da força do atirador. O Narrador pode escolher modificar tanto a dificuldade quanto os dados de dano se sentir que o personagem está fora do alcance efetivo da arma.

\subparagraph{\bf Cobertura:}
A cobertura atrapalha as tentativas do oponente de atirar no personagem, mas isso também prejudica a capacidade desse personagem de responder ao fogo. Veremos a seguir os tipos básicos de cobertura e os modificadores que elas impõem à dificuldade do atacante. Esses modificadores também se impõem à resposta ao fogo, embora em menor grau. O personagem que responde ao fogo subtrai um ponto desses modificadores. Portanto, o personagem que responde ao fogo de trás de uma parede acrescenta um a sua dificuldade, enquanto um personagem deitado no chão não sofre impedimento.
\begin{itemize}[noitemsep]
\item Cobertura Simples: Deitado no chão +1
\item Cobertura Parcial: Atrás de uma parede +2
\item Cobertura Média: Só a cabeça exposta +3
\end{itemize}

\subparagraph{\bf Mirar:}
O personagem que passar algum tempo mirando pode atirar com muito mais precisão do que outro que simplesmente improvisa um disparo. Entretanto, para mirar adequadamente, o personagem só poderá se deslocar a passo lento, e o alvo deverá permanecer o tempo todo no campo de visão do personagem.
Para cada turno dedicado à mirar, o jogador acrescentará um dado a sua parada de Destreza + Armas de Fogo, até um máximo igual ao nível de Percepção do personagem. Uma mira telescópica acrescentará dois dados a mais à parada. Contudo, esse bônus só se aplicará a um tiro por vez. O personagem com uma mira telescópica e um nível de Percepção 3 poderia passar três turnos mirando e ganhar cinco dados adicionais para o teste (dois pela mira telescópica e três por mirar). Para conseguir o bônus novamente, ele terá de passar outros três turnos mirando.
O personagem precisa ter Armas de Fogo para receber este benefício

\subparagraph{\bf Modo Automático:}
Algumas armas de fogo permitem ao usuário esvaziar um pente inteiro em questões de segundos. Disparar uma arma no modo automático acrescenta dez dados à jogada de ataque, mas isso eleva a dificuldade em dois pontos, pois o recuo estraga a mira do personagem. Este ataque só é permitido se o pente estiver cheio no mínimo até a metade. Naturalmente, após este ataque o pente estará completamente vazio. 
O personagem também pode optar por metralhar uma área (efeito "mangueira") em vez de esvaziar o pente num alvo. O sistema é o mesmo do modo automático, com a diferença que os sucessos são distribuídos igualmente entre todos os alvos. Se o personagem atirar numa quantidade de alvos maior que o número de sucessos conseguidos pelo jogador, o Narrador escolherá quais alvos serão atingidos.

\subparagraph{\bf Movimento:}
Mover-se ao atirar ou abrir fogo contra um alvo em movimento a uma velocidade superior à de caminhada eleva a dificuldade em +1.

\subparagraph{\bf Rajada Curta:}
Algumas armas são capazes de disparar três balas toda vez que o personagem puxa o gatilho. Fazer isso em combate acrescenta três dados ao teste de ataque, mas eleva a dificuldade em um ponto. Obviamente, atirar com essa cadência a tabela de Armas de Alcance para ver quais armas são capazes de disparar uma rajada curta.

\subparagraph{\bf Recarregar:}
As armas que usam pentes de balas podem ser recarregadas rapidamente em combate desde que o personagem tenha preparado um pente sobressalente. A arma pode ser recarregada e disparada no mesmo turno. O jogador simplesmente perde dois dados de sua parada de ataque para compensar o tempo perdido na recarga.
Um revolver pode ser recarregado dessa maneira somente com um carregador apropriado. Se tiver de recarregar um revólver manualmente, o personagem levará um turno inteiro e a operação exigirá sua total concentração, mas poderá ser executada sem um teste se o personagem tiver pelo menos um ponto de Armas de Fogo. Recarregar um pente (colocar de fato balas no pente), entretanto, requer um teste de Destreza + Armas de Fogo (dificuldade 6). Apenas um sucesso é necessário, mas isso leva o turno inteiro.

\subparagraph{\bf Tiros múltiplos:}
O jogador deve realizar uma ação múltipla ou usar um ponto de Fúria para disparar vários tiros num único turno. As rajadas curtas e o modo automático contam como um "tiro" para essa finalidade. O número máximo de tiros que podem ser disparados por turno é igual ao da cadência de tiro da arma (listada na tabela).
