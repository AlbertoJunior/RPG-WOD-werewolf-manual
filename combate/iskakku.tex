\titulo{Iskakku}

\subsection{Iskakku}

\subparagraph*{\bf Pré-requisitos:} Destreza •••, Raciocínio •••, Armas Brancas ••

{\bf MET:} Iskakku precisa ser comprado como uma Habilidade normal e separada. Aqueles que possuem Armas Brancas com especialização em Bastões podem substituir sua Habilidade Armas Brancas (menos dois níveis) por Iskakku durante os retestes. Cada manobra é marcada por quantos níveis de Iskakku precisa-se aprender antes de tentar usá-la. Essas manobras são ganhas quando ele compra o próximo nível de Iskakku.

\subparagraph{\bf • Bloquear:}
O usuário de Iskakku tem como sua principal ação a defesa, sua ou de seus companheiros, por este motivo, treinam bastante as formas de utilizar o bastão como uma ferramenta de proteção e não de ataque. Devido a isto, o usuário de Iskakku ganha +1 dado em qualquer ação defensiva.

\subparagraph{\bf •• Proteger:}
As habilidades de defesa de seu personagem faz com que ele antecipe alguns ataques, principalmente os que não são direcionados a ele. Existem histórias de um único usuário conseguir proteger toda sua matilha sozinho, evitando que eles sejam atacados. O seu personagem pode usar uma ação de defesa, onde utiliza o bastão para Aparar, Bloquear ou Desequilibrar (dificuldade +1) o ataque de seu oponente e assim consegue proteger qualquer aliado próximo, mesmo que sua iniciativa seja a ultima, ele pode mudar uma ação de ataque para uma defesa, se for para defender um aliado sem o consumo da Força de Vontade. Ao desequilibrar, a cada 2 sucessos acima do ataque do oponente, este perde 1 dado na sua próxima ação.
\begin{itemize}[noitemsep]
\item Teste: Destreza + Iskakku
\item Dificuldade: 6 (+1 para Desequilibrar)
\item Dano: Nenhum
\item Ações: 1
\end{itemize}

\subparagraph{\bf •• Masahu Qatu:}
(Atingir a Mão) Quando estiver sendo atacado por alguém usando uma arma branca, o personagem pode optar por aparar os golpes, o que também serve como uma tentativa de deslocar o pulso do atacante. Isto não apenas desvia o golpe que está vindo como também potencialmente incapacita seu oponente. O defensor faz uma jogada de Destreza + Iskakku (dificuldade 7 *). Se o número de sucessos exceder os sucessos do oponente, então o ataque não apenas é desviado, mas o pulso do atacante se desloca e ele não pode mais usar aquela mão até que ele se cure (o equivalente a curar um nível de dano de contusão).
\begin{itemize}[noitemsep]
\item Teste: Destreza + Iskakku
\item Dificuldade: 7*
\item Dano: Especial
\item Ações: Especial
\end{itemize}

\textit{{\bf OBS.: } * modificação na dificuldade em -1.}

\subparagraph{\bf •• Tammabukku Istu Kur:}
(O Dragão Emerge da Montanha) Como uma súbita saída da usual série de giros, o personagem subitamente impulsiona o bastão para frente como normalmente se faria com uma lança. Este ataque é mirado na face e não pode ser tentado mais que uma vez seguida. 
\begin{itemize}[noitemsep]
\item Teste: Destreza + Iskakku
\item Dificuldade: 6
\item Dano: Arma + 2
\item Ações: 1
\end{itemize}

\subparagraph{\bf •• Tabaku Kur:}
(Tirar o Solo) Com um movimento circular do bastão, o atacante pode tentar fazer seu oponente tropeçar, forçando-o ao chão. O atacante testa Destreza + Iskakku (dificuldade 7*) enquanto o defensor resiste com Destreza + Esportes (dificuldade 7). Se o atacante tiver mais sucessos, então o defensor cai. De outra forma ele continua em pé. 
\begin{itemize}[noitemsep]
\item Teste: Destreza + Iskakku
\item Dificuldade: 7*
\item Dano: Especial
\item Ações: 1
\end{itemize}

\subparagraph{\bf ••• Isten Kima Ummanate:}
(Um Como Um Exército) O bastão pode ser empunhado de tal forma que ambas as extremidades possam ser usadas para atacar em rápida sucessão. Quando usar a técnica, o personagem divide sua parada de dados pela metade, usando a primeira (e maior, caso não seja uma divisão igual) parada para seu ataque inicial e a segunda para outro ataque. Enquanto esses ataques não são tão ameaçadores como um ataque completo, ele tende a forçar o oponente a se defender da rajada de golpes apenas para evitar ser atingido. Todo o dano causado com esses ataques é de um dado a menos.
\begin{itemize}[noitemsep]
\item Teste: Destreza + Iskakku
\item Dificuldade: 6
\item Dano: Arma - 1
\item Ações: 1
\end{itemize}

\subparagraph{\bf •••• Sepu Istu An:}
(Pé Vindo do Céu) O personagem usa seu bastão para jogar a si mesmo pelo ar e desfere um poderoso chute em seu oponente. O atacante primeiro precisa ter um espaço para correr, então testa sua Força + Esportes (dificuldade 5) para determinar se ele pode pular longe o suficiente até seu oponente. A distância pulada é de 2,50m por sucesso. Ao aterrissar ele testa Destreza + Briga (dificuldade 7) para desferir um poderoso chute em seu oponente. As regras padrões de combate se aplicam a este ataque.
\begin{itemize}[noitemsep]
\item Teste: Destreza + Briga
\item Dificuldade: 7
\item Dano: Força + 3
\item Ações: 2
\end{itemize}

\subparagraph{\bf •••• Sepsu Sepu:}
(Pé Poderoso) Em adição ao ataque extra permitido pelo Isten Kima Unammate, o personagem pode agora dividir seus dados de ataque em três paradas e usar o terceiro para um chute. Usar todos os três ataques é normalmente reservado para lutas contra vários combatentes sem treinamento. 

\subparagraph{\bf ••••• Adannu Lukur Daku:}
(Hora Marcada para o Fim do Inimigo) O mestre de Iskakku geralmente termina uma luta antes que ela realmente comece. Ele fica em espera pelo primeiro ataque de seu oponente. Ele bloqueia com uma extremidade do bastão enquanto move-se passando pelo oponente e então desfere um ataque poderoso na parte de trás da cabeça com a outra extremidade. Se desferido com precisão, a pancada no cerebelo é suficiente para deixar qualquer um inconsciente. O defensor primeiro precisa fazer um bloqueio bem-sucedido (dificuldade 7) e então faz uma jogada de Destreza + Iskakku (dificuldade 8) para atacar. Se sucedido, o dano causado não é aplicado como dano nos Níveis de Vitalidade, mas é comparado com o Vigor do oponente. Se o dano for maior que o Vigor, o oponente fica inconsciente. Este dano imaginário não pode ser absorvido.
\begin{itemize}[noitemsep]
\item Teste: Destreza + Iskakku
\item Dificuldade: 8
\item Dano: Arma
\item Ações: 2
\end{itemize}