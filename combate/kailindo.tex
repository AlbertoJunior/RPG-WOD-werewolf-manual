\titulo{Kailindô}

\subsection{Kailindô}

\subparagraph*{\bf Pré-requisitos:} Destreza •••, Vigor •••, Briga ••

\subsection{\bf Manobras Padrão}

\subparagraph{\bf • Dança da Tempestade:}
Essa manobra não é um ataque, é mais uma técnica de intimidação usada para desencorajar os oponentes. O Kailindorani executa uma série de manobras, acompanhadas de uma linguagem corporal ameaçadora. O resultado é uma demonstração de que o lutador é gracioso e predatório ao mesmo tempo. O jogador testa Manipulação + Kailindô, resistido pelo Raciocínio + Briga do oponente; esta é uma ação de um turno inteiro. Se o jogador vencer o teste, a dificuldade de seus ataques contra seu oponente é reduzida em 1 por um turno por sucesso em que ele exceder os sucessos do adversário — caso venha a golpear. Essa manobra pode ser usada apenas uma vez por oponente por combate.
\begin{itemize}[noitemsep]
\item Teste: Manipulação + Kailindô 
\item Dificuldade: 7
\item Dano: Nenhum 
\item Ações: Especial
\end{itemize}

\subparagraph{\bf • Pequeno Ciclone:}
O Kailindorani se abaixa com sua perna estendida, tentando varrer as pernas do oponente e derrubá-lo. O oponente pode resistir com um teste de Destreza + Esquiva contra uma dificuldade 6, o Kailindorani será bem sucedido se ele obtiver mais sucessos que o oponente. O dano dessa manobra é igual a um de dano +1 por sucesso extra.
\begin{itemize}[noitemsep]
\item Usável por: Hominídeo / Crinos
\item Teste: Destreza + Kailindô 
\item Dificuldade: 6
\item Dano: 1 
\item Ações: 1
\end{itemize}

\subparagraph{\bf • Turbilhão:}
Essa técnica defensiva exige que o Kailindorani movimente seus braços diante de si como o vento. Cada sucesso no teste adiciona um dado para cada tentativa de manobra de bloqueio durante o mesmo turno da técnica Turbilhão. O Kailindorani não pode atacar, correr ou fazer qualquer coisa exceto bloquear no mesmo turno que ele usar Turbilhão.
\begin{itemize}[noitemsep]
\item Teste: Destreza + Kailindô 
\item Dificuldade: 7
\item Dano: Nenhum 
\item Ações: 1
\end{itemize}

\subparagraph{\bf •• Vento Ilusório:}
Saltando em direção a seu oponente, o Kailindorani inicialmente finta um chute frontal, mas ao invés disso ele passa pelo seu oponente e o golpeia pelas costas durante a passagem. Essa manobra pode ser usada com uma mudança de forma e não pode ser bloqueada ou aparada, apenas esquivada. 
\begin{itemize}[noitemsep]
\item Usável por: Hominídeo / Crinos
\item Teste: Destreza + Kailindô
\item Dificuldade: 5
\item Dano: Força 
\item Ações: 1
\end{itemize}

\subparagraph{\bf ••• Chute Tornado:}
O Kailindorani gira como um tornado, com uma velocidade incrível, adicionando força cinética à força de seu chute.
\begin{itemize}[noitemsep]
\item Usável por: Hominídeo / Crinos
\item Teste: Destreza + Kailindô 
\item Dificuldade: 7
\item Dano: Força + 2 
\item Ações: 1
\end{itemize}

\subparagraph{\bf ••• Vento da Submissão:}
Agarrando o pulso de seu oponente no momento exato, o Kailindorani imobiliza seu inimigo forçando-o para baixo. O Kailindorani precisa obter um número de sucessos em seu teste de ataque maior que o nível de Destreza do oponente para ser bem-sucedido nessa façanha. Um oponente subjugado pode tentar um teste resistido de Destreza (dificuldade 8 para ambos os personagens) para se libertar ou um segundo teste resistido de Força (dificuldade 8 para o personagem preso, 6 para o Kailindorani, visto que o apresamento é mais questão de jeito). O dano é determinado pela perícia do Kailindorani, não pela Força, como é de costume, e não pode ser agravado. 
\begin{itemize}[noitemsep]
\item Usável por: Hominídeo / Crinos
\item Teste: Destreza + Kailindô 
\item Dificuldade: 6
\item Dano: Kailindô
\item Ações: 1
\end{itemize}

\subparagraph{\bf •••• Tempestade Cadente:}
O Kailindorani tenta agarrar a garganta de seu oponente com suas pernas e braços enquanto lança a si mesmo diretamente sobre seu alvo e mergulha-o no chão enquanto o estrangula. Para conseguir isto, o Kailindorani precisa obter mais sucessos que o nível de Força de seu oponente em seu teste de ataque. Um personagem estrangulado perde um Nível de Vitalidade por turno, e para escapar precisa vencer um teste resistido de Força com dificuldade 6. Dano por sufocamento não pode ser absorvido, mas é curado após uma hora de repouso.
\begin{itemize}[noitemsep]
\item Usável por: Hominídeo-Lupino
\item Teste: Destreza + Kailindô 
\item Dificuldade: 8
\item Dano: Especial
\item Ações: 1
\end{itemize}

\subparagraph{\bf ••••• Vento Poderoso:}
O Kailindorani corre frente ao seu oponente e tenta nocauteá-lo com um chute na cabeça ou no torso. Se o Kailindorani causar mais Níveis de Vitalidade de dano que o nível de Força do oponente, o oponente é nocauteado.
\begin{itemize}[noitemsep]
\item Usável por: Hominídeo / Crinos
\item Teste: Destreza + Kailindô
\item Dificuldade: 8
\item Dano: Força + 1 
\item Ações: 2
\end{itemize}

\subsection{\bf Manobras Metamórficas}

\subparagraph{\bf •• Brisa Desvanecida:}
O Kailindorani muda de uma forma menor para uma maior à medida que ele lentamente recua para socar, chutar ou rasgar seu oponente. Isso aumenta a dificuldade do ataque do oponente em 1. 
\begin{itemize}[noitemsep]
\item Usável por: Todas, exceto Hispo e Crinos
\item Teste: Destreza + Kailindô 
\item Dificuldade: 7
\item Dano: Força 
\item Ações: 1
\end{itemize}

\subparagraph{\bf •• Brisa Mutável:}
Passando de uma forma maior para uma menor, o Kailindorani se evade de um ataque eminente. Cada sucesso nesse teste adiciona 1 à dificuldade de ataque do oponente, até um limite de 10. Se o Kailindorani gastar um ponto de Fúria para mudar de forma, ele adiciona um sucesso automático ao seu teste da manobra Brisa Mutável.
\begin{itemize}[noitemsep]
\item Usável por: Todas, exceto Hominídea e Lupina
\item Teste: Destreza + Kailindô 
\item Dificuldade: 7
\item Dano: Nenhum 
\item Ações: 1
\end{itemize}

\subparagraph{\bf •• Tempestade Crescente:}
Agarrando e apresando seu oponente enquanto muda para uma forma mais forte, o Kailindorani pode usar sua força crescente para esmagá-lo durante a mudança. Para se libertar o oponente precisa vencer um teste resistido de Força versus Força. O dano causado por esta manobra não pode ser agravado, a menos que o Garou mude para Hispo (onde ele pode estar apresando com suas mandíbulas). Essa manobra exige o gasto de um ponto de Fúria.
\begin{itemize}[noitemsep]
\item Usável por: Todas, exceto Hispo e Crinos
\item Teste: Destreza + Kailindô 
\item Dificuldade: 7
\item Dano: Força + 1 
\item Ações: 1
\end{itemize}

\subparagraph{\bf ••• Furacão:}
O Kailindorani muda para Crinos enquanto arremessa a si mesmo contra o oponente, adicionando o ponto de apoio e força da forma de batalha à sua ação. O oponente é arremessado a dois metros por sucesso + o nível de Força do Kailindorani. O dano é normalmente Força + sucessos, mas dependendo de onde o oponente caia, pode ser maior ou menor, a critério do Narrador.
\begin{itemize}[noitemsep]
\item Usável por: Todas, exceto Crinos
\item Teste: Destreza + Kailindô 
\item Dificuldade: 7
\item Dano: Especial 
\item Ações: 1
\end{itemize}

\subparagraph{\bf ••• Tempestade Nascente:}
Esta manobra foi desenvolvida para derrubar os oponentes mais fortes. O Kailindorani muda para uma forma maior e mais forte enquanto soca, chuta ou golpeia o oponente com as garras.
\begin{itemize}[noitemsep]
\item Usável por todas as formas exceto Crinos
\item Teste: Destreza + Kailindô	
\item Dificuldade: 5
\item Dano: Força + 2	
\item ações: 1
\end{itemize}

\subparagraph{\bf ••• Vento Liquefeito:}
Mudando para uma forma menor enquanto se está sendo preso ou agarrado, o Kailindorani pode facilmente escapar de seu oponente. Teste Destreza + Kailindô normalmente e adicione um sucesso à sua parada de Força para propósitos de escapar do apresamento. Uma falha crítica causará dano adicional ao apresamento, bem como no aumento da dificuldade de tentar escapar novamente em 1.
\begin{itemize}[noitemsep]
\item Usável por: Todas, exceto Hominídea e Lupina
\item Teste: Destreza + Kailindô 
\item Dificuldade: 6
\item Dano: Nenhum 
\item Ações: 1
\end{itemize}

\subparagraph{\bf •••• Agitação Súbita:}
Mudando para uma forma maior enquanto se está sendo agarrado, o Kailindorani toma vantagem da situação arremessando seu oponente ao chão, por uma distância de 30cm por sucesso + 30cm por nível de Força do Kailindorani.
\begin{itemize}[noitemsep]
\item Usável por: Todas, exceto Hispo e Crinos
\item Teste: Destreza + Kailindô 
\item Dificuldade: 6
\item Dano: Especial 
\item Ações: 1
\end{itemize}

\subparagraph{\bf •••• Mudança Habilidosa:}
O Kailindorani executa um golpe de carga em seu oponente na forma Lupina. Enquanto seu oponente cai, o Kailindorani muda para uma forma maior e ganha a vantagem no combate. Isso reduz todas as dificuldades dos ataques do Kailindorani contra seu oponente no turno subsequente em 1. Essa tática é muito útil quando vários personagens trabalham juntos como uma matilha. 
\begin{itemize}[noitemsep]
\item Usável por: Lupino
\item Teste: Destreza + Kailindô 
\item Dificuldade: 6
\item Dano: Nenhum 
\item Ações: 1
\end{itemize}

\subparagraph{\bf ••••• Golpear o Vento:}
O Kailindorani sofre um golpe, mas muda para uma forma maior e desfere um contra-ataque. Embora ele esteja sendo golpeado, a dificuldade do teste de absorção é reduzida em 2. Essa manobra exige o gasto de um ponto de Fúria.
\begin{itemize}[noitemsep]
\item Usável por: Todas, exceto Crinos
\item Teste: Destreza + Kailindô 
\item Dificuldade: 5
\item Dano: Normal 
\item Ações: 1
\end{itemize}

\subparagraph{\bf ••••• Brisa Escorregadia:}
Durante uma esquiva, o Kailindorani assume uma forma menor. O jogador pode adicionar os sucessos do teste de mudança de forma à sua parada de esquiva pelo resto do turno. Se o Kailindorani gastar um ponto de Fúria ao invés de testar Vigor + Instinto Primitivo, conta como se ele tivesse obtido 5 sucessos.
\begin{itemize}[noitemsep]
\item Usável por: Todas, exceto Hominídea e Lupina 
\item Teste: Especial 
\item Dificuldade: 6 
\item Dano: nenhum 
\item Ações: 1
\end{itemize}