\tituloM{Correntes e Chicotes}{Armory}{0.1}{Alberto}

\subsection{\bf Combate com Correntes e Chicotes}

\subparagraph*{\bf Pré-requisitos:} Destreza •••, Armas Brancas ••• 

\subparagraph*{Quando referido a corrente, não significa que seja só com a corrente, mas sim com Corrente e Chicote.}

\subsection{\bf Manobras}

\subparagraph{\bf • Defesa Impenetrável:} 
Você pode optar por não utilizar uma ação de ataque e ao invés disso, girar ou torcer a corrente de forma defensiva, ganhando assim +2 dados de defesa para aparar ou bloquear ataques recebidos. Essa manobra também permite fazer isso contra vários oponentes sem sofrer penalidades, podendo ser defendido até 3 ataques. O primeiro e o segundo ataque não causam modificadores negativos de defesa.

\subparagraph{\bf •• Prender a Mão:} 
Esta manobra defensiva é realizada contra um ataque recebido (Briga ou Armas Brancas). Quando atacado, seu personagem envolve o braço do atacante com a corrente, agarrando-o com um teste de Força + Armas Brancas. A defesa do inimigo não é afetada (a critério do narrador), mas seus sucessos na rolagem de ataque são. Caso seus sucessos superem o do atacante, este fica preso à corrente e pode tentar escapar no próximo turno com um teste de Força + Briga. Se o inimigo tentar atacar novamente sem se livrar da corrente, seu ataque ainda é diminuído pelos sucessos que você obteve. Essa manobra deve ser declarada na ação do atacante e somente caso você tenha uma iniciativa maior e a realização desta ação significa que seu personagem não pode fazer um ataque neste turno.
\begin{itemize}[noitemsep]
\item Teste: Força + Armas Brancas
\item Dificuldade: 6
\item Dano: Nenhum
\item Ações: 1
\end{itemize}

\subparagraph{\bf ••• Laçar:}
Suas habilidades com as correntes já atingiu níveis fantásticos. Você pode laçar qualquer membro de um oponente fazendo um teste de Destreza + Armas Brancas, caso ele não consiga desviar ou aparar estará preso. Para se libertar ele deve gastar uma ação e fazer um teste resistido de Força contra Força, caso não consiga, sofrerá uma penalidade no turno de +1 a +3, a depender do local que esteja laçado e da ação que deseje fazer. Uma pessoa não consegue correr tendo seus pés laçados, sofrendo assim uma penalidade de +3, mas caso seja só um dos pés, a penalidade é de +1, mas esta penalidade está sempre a critério do narrador.
\begin{itemize}[noitemsep]
\item Teste: Destreza + Armas Brancas
\item Dificuldade: 7
\item Dano: Nenhum
\item Ações: 1
\end{itemize}

\subparagraph{\bf •••• Estrangulamento:}
Seu personagem tenta enforcar o oponente, testando Força + Armas Brancas. A vítima pode tentar libertar-se em sua próxima ação testando Força + Briga, e deve obter em sucessos a sua Força + 1 para se livrar. Esta manobra não causa dano ou seja, não pode matar o oponente, serve apenas para deixa-lo inconsciente, tensionando a corrente para asfixiá-lo. Caso bem sucedido, a vitima começa a sufocar no turno seguinte. Para cada turno que a vitima não conseguir se soltar, ela sofre uma penalidade cumulativa de -1 em todas as rolagens para resistir. Quando a penalidade for igual ao Vigor da vítima, ele cai inconsciente. Essa manobra, quando completa, causa um único ponto de dano na vítima. Essa manobra de combate é ineficaz contra personagens que não precisam respirar.
\begin{itemize}[noitemsep]
\item Teste: Força + Armas Brancas
\item Dificuldade: 7
\item Dano: Especial
\item Ações: 1
\end{itemize}

\subparagraph{\bf ••••• Girar e Lançar:}
Seu personagem neste nível é altamente habilidoso em usar correntes e pode fazer ataques focalizados com precisão. Ao girar a corrente algumas vezes, você pode criar impulso para um único ataque certeiro. Em um ataque direcionado, você pode ignorar até -2 de penalidades associadas a ataques localizados. Em outras palavras, os ataques ao torso ou aos membros de um oponente são feitos sem penalidade, os ataques direcionados a cabeça terão uma penalidade de apenas -1, na mão -2 e no olho -3. Está manobra também pode ser utilizada em conjunto com Laçar diminuindo a dificuldade em -2 quando focado em um único membro e em -1 para os dois tornozelos ou torso, gastando 2 ações.

{\bf Desvantagem:} Seu personagem não pode ter feito ou fazer qualquer ação Defensiva no turno.