\titulo{Manobras Especiais}

\subparagraph{\bf Ação Evasiva:}
Semelhante à esquiva, desviar-se de um oponente envolve saltar, rodopiar e geralmente manter-se um passo à frente dele. Esta manobra não inflige dano, mas cada sucesso obtido subtrai um sucesso no teste de ataque de um opoente. Se o jogador que interpreta o personagem que se desvia obtiver mais sucessos que o atacante, este não apenas errará o ataque como o primeiro vai se colocar em boa posição para contra-atacar. O personagem que se desvia receberá um bônus igual a -1 sobre a dificuldade de seu ataque no turno seguinte, supondo-se que ele venha a agir primeiro.
Ao contrário da esquiva, o personagem não pode abortar uma ação e mudar para uma ação evasiva. Esta deve ser sua ação declarada.
{\bf Usável por:} Todas as formas.
\begin{itemize}[noitemsep]
\item Teste: Raciocínio + Esquiva
\item Dificuldade: 6
\item Dano: Nenhum
\item Ações: 1
\end{itemize}

\subparagraph{\bf Escárnio:}
O Garou brinca com o oponente, rosnando, mostrando os dentes e proferindo insultos. Está tática pode alarmar ou distrair o alvo e fazê-lo hesitar, o que dá ao Garou uma vantagem. A cada dois sucessos que o jogador obtiver num teste de Manipulação + Intimidação (no caso de oponentes não-Garou) ou Expressão (com outros Garou), o oponente perderá um dado em sua ação seguinte.
Esta manobra pode ser usada por uma matilha (veja a seção Táticas de Matilha). Se isso acontecer, os efeitos serão cumulativos, o que significa que a parada de dados de um oponente pode ser reduzida a zero, Se assim for, ele não poderá realizar nenhuma ação a não ser se esquivar.
O Garou submetido a esta manobra, principalmente se realizada por uma matilha, pode entrar em frenesi. Um teste de Fúria deve ser feito e a dificuldade será reduzida em um ponto se a provocação partir de uma matilha.
{\bf Usável por:} Todas as formas.
\begin{itemize}[noitemsep]
\item Teste: Manipulação + Expressão/Intimidação
\item Dificuldade: Raciocínio + 4 do opoente
\item Dano: Nenhum
\item Ações: 1
\end{itemize}

\subparagraph{\bf Incapacitar:}
Com este ataque violento, o lobisomem enterra as presas na parte inferior da perna de seu alvo e arranca os tendões. Se tiver sucesso, o ataque tolherá severamente os adversários quadrúpedes e aleijará os bípedes (diminua pela metade as taxas de movimentação dos adversários quadrúpedes). Também é possível executar este ataque com as garras, embora isso pareça menos natural.
O dano provocado por este ataque é agravado. Em geral, um lobisomem solitário usa esta manobra para retardar o oponente até que sua matilha possa se juntar à batalha.
{\bf Usável por:} Crinos / Lupina
\begin{itemize}[noitemsep]
\item Teste: Destreza + Briga
\item Dificuldade: 8
\item Dano: Força + Aleijamento
\item Ações: 1
\end{itemize}

\subparagraph{\bf Mandíbula de Ferro:}
O Garou cerra as mandíbulas no pescoço de um alvo, não para matar, e sim para imobilizar. Este ataque só pode ser executado por trás ou por cima de um oponente, de modo que o atacante possa usar todo o peso de seu corpo a seu favor. O atacante deve primeiro ter sucesso numa mordida, com uma penalidade igual a +1 sobre a dificuldade. Em vez de lançar os dados para avaliar o dano, entretanto, o atacante e o defensor deve ambos fazer um teste resistido de Força + Esportes. Se o atacante vencer, ele forçará o alvo contra o chão e o manterá assim. Se perder, o atacante não conseguirá imobilizar o alvo, mas a mordida poderá infligir dano normalmente.
O personagem imobilizado pode tentar escapar em sua próxima ação. O jogador que o interpreta deve testar Força + Briga (dificuldade igual a Briga +4 do oponente) num teste resistido contra Força + Briga do atacante (dificuldade igual a Briga + 2 do defensor). Se falhar, o defensor permanecerá imobilizado. Ele escapará se conseguir igualar o número de sucessos do atacante, mas receberá uma quantidade de dano igual ao número de sucessos do atacante (que ele poderá absorver). Se obtiver mais sucessos que o atacante, ele escapará sem maiores danos.
{\it Exemplo: A luta "de brincadeirinha" entre Presa e Caminha já não é tão de brincadeira. Cansado do espernear de seu oponente, Presa prende Caminha numa mandíbula de ferro. Caminha não gosta dessa reviravolta nos acontecimentos e tenta escapar. Ambos os jogadores fazem teste de Força + Briga por seus personagens. Os dois têm Briga 4 e, portanto, Presa tem uma dificuldade igual a 6 (Briga + 2 de Caminha) e Caminha tem uma dificuldade igual a 8 (Briga + 4 de Presas). Os dois jogadores obtêm três sucessos. Caminha escapa, mas tem de absorver três níveis de dano agravado. Ele o faz com facilidade e decide dar fim à luta-treino antes que as coisas piorem.}
{\bf Usável por:} Crinos, Hispo e Lupinos
\begin{itemize}[noitemsep]
\item Teste: Destreza + Briga
\item Dificuldade: 6
\item Dano: Especial
\item Ações: 1
\end{itemize}

\subparagraph{\bf Salto Dilacerante:}
Uma manobra de luta para os Garou um pouco mais graciosos e leves, o salto dilacerante envolve passar por um oponente com um salto e golpeá-lo em meio ao movimento. Se bem-sucedida, a manobra faz o Garou aterrissar longe do alcance de seu oponente. 
O narrador deve primeiro verificar de quantos sucessos o jogador precisará num teste de Destreza + Esportes (dificuldade 3; as distâncias estão indicadas na tabela de Saltos). Se o jogador não obtiver um número de sucessos suficiente para fazer seu personagem passar pelo oponente, o personagem aterrissará onde a tabela assim o indicar e poderá ainda usar sua ação garantida por Fúria, submetida a uma penalidade igual a +1 sobre a dificuldade. Se tiver sucesso, ele então deverá fazer um teste de Destreza + Briga para o ataque com as garras. Se o ataque com as garras falhas, o personagem ainda aterrissará onde planejou. 
Este ataque provoca ferimentos agravados. Pode também ser tentado com um soco (provocando dano por contusão) ou com uma arma (provocando dano de acordo com a arma e transformando o teste em Destreza + Armas Brancas).
{\bf Usável por:} Glabro e Crinos
\begin{itemize}[noitemsep]
\item Teste: Destreza + Esportes / Destreza + Briga
\item Dificuldade: 8
\item Dano: Força +1
\item Ações: 2
\end{itemize}