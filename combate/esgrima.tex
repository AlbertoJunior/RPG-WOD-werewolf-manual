\tituloM{Esgrima e Kendo}{Armory}{0.1}{Alberto}

\subsection{\bf Combate com Espadas}

\subparagraph*{\bf Pré-requisitos:} Destreza/Raciocínio •••, Armas Brancas •••

\subsection{\bf Manobras}

\subparagraph{\bf • March/Nuki Waza (Marcha):} 
A marcha é a manobra base da esgrima, consiste em avançar ou recuar, com o objetivo de manter uma distância do adversário, ou seja, seu personagem evita que o oponente o alcance e está sempre pronto para atacar. O seu personagem está mais preparado para o combate, ganha +2 na iniciativa e quando esta manobra for utilizada com uma ação ganha um bônus de +3 dados para esquivar ou para atacar.

\subparagraph{\bf •• En Garde/Suriage Waza (Em Guarda):} 
Ao executar esta manobra o seu personagem assume uma postura defensiva, as pernas ficam levemente flexionadas o pé dominante a frente e alinhado com o outro e distribuindo seu peso, aumentando assim as chances de conseguir aparar um ataque, quanto mais precisa for a guarda, maior será o bônus ganho. Por exemplo, se o personagem assume uma postura para defender sua oitava e recebe um ataque justamente nesse ponto, ele ganha um bônus de +3 dados, caso o ataque recebido seja apenas do mesmo lado o bônus é de +1, caso a guarda seja efetuada para o lado que não seja o alvo, nenhum bônus é ganho.

{\bf As guardas:}
\begin{itemize}[noitemsep]
\item A guarda de quarta: defende o tronco superior esquerdo.
\item A guarda de sexta: defende o tronco superior direito.
\item A guarda de sétima: defende o tronco inferior esquerdo.
\item A guarda de oitava: defende o tronco inferior direito.
\end{itemize}

\subparagraph{\bf •• Coup/Kaburi (Estocada):} 
A estocada é um ataque simples, mas poderoso. A postura de um esgrimista (uma perna que ancora a posição do seu personagem e a outra perna pulando para frente) dá mais força a este ataque. Quando seu personagem faz um ataque de estocada, mergulhando a lâmina em direção a um oponente, ele faz isso com um bônus de +1 dado.

\subparagraph{\bf ••• Riposte/Uchiotoshi Waza (Ripostar):}
Para executar um Ripostar seu personagem deve ser atacado. Ele sai do caminho do ataque usando sua esquiva e enquanto o oponente está com a guarda aberta, ele pode fazer um ataque repentino e rápido, que é executado com sua próxima ação. No entanto, o oponente não pode defender este ataque.

{\bf Desvantagem:} Se o seu oponente sofre muitos ataques como esse ou já o viu utiliza-lo muitas vezes, ele tem seu efeito reduzido ou até mesmo anulado, ou seja, só pode ser usado até 6 - Raciocínio do oponente.
\begin{itemize}[noitemsep]
\item Teste Primário: Destreza + Esquiva
\item Teste Secundário: Destreza + Armas Brancas (com -1 dado)
\item Dificuldade: 6
\item Dano: Arma
\item Ações: 2
\end{itemize}

\subparagraph{\bf •••• Feinte/Kiai (Fintar):} 
Seu personagem sabe como simular um ataque falso destinado a confundir seu oponente. Faça uma rolagem normal de ataque testando (Destreza + Armas Brancas), este é um teste resistido com o oponente. Esse ataque é falso e não atinge o inimigo ou causa qualquer dano. Se o seu personagem atingir mesmo um único sucesso, no entanto, o oponente fica momentaneamente confuso e perde o equilíbrio, não podendo se defender contra qualquer outro próximo ataque direcionado a ele (sendo um ataque seu ou de qualquer outra coisa). Se falhar a ação é perdida e caso seja uma falha crítica, seu personagem ficou desajeitado e abriu totalmente sua guarda, impossibilitando de se defender do próximo ataque.
\begin{itemize}[noitemsep]
\item Teste: Destreza + Armas Brancas
\item Dificuldade: 7
\item Dano: Nenhum
\item Ações: 1
\end{itemize}

\subparagraph{\bf ••••• Moulinet/Nidan Waza (Carretel):}
Se o seu personagem acerta o adversário com sua espada, ele pode rodar o pulso e executar um corte rápido em espiral com a ponta da arma. Este corte adicional não requer uma rolagem adicional. O corte faz danos letais ao adversário igual à sua Destreza, ou seja, o personagem ganha dados de dano igual seu nível atual de Destreza e testa eles com dificuldade 4.
{\bf Desvantagem:} Para realizar essa manobra, o personagem deve gastar um ponto de Força de Vontade antes de executar o seu ataque e declarar que é para a manobra. A Força de Vontade não lhe concede o sucesso automático. Se o ataque falhar, o ponto de Força de Vontade é perdido e a manobra não é efetuada.