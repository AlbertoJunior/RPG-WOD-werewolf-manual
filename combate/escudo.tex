\tituloM{Escudos}{Armory}{0.1}{Leno Oliveira, Fabiano Fonseca}

\subsection{\bf Combate com Escudo}

\subparagraph*{\bf Pré-requisitos:} Força ••, Destreza ••, Vigor ••, Armas Brancas •••

\subsection{\bf Manobras}

\subparagraph{\bf • Defender Aliado:}
O uso de escudo significa proteger a si mesmo, mas mais do que isso, a proteger seus aliados. A personagem aprende desde cedo a utilizar um escudo para bloquear ataques que não sejam focados nela, desde que esteja próxima o suficiente para proteger seu aliado.
\begin{itemize}[noitemsep]
\item Teste: Destreza + Armas Brancas + Bônus do Escudo
\item Dificuldade: 6
\item Dano: Nenhum
\item Ações: 1
\end{itemize}

\subparagraph{\bf •• Ataque com Escudo:}
O uso de escudo da personagem se integra perfeitamente com o de sua arma. Ela pode descansar a arma por cima ou pelo lado de um escudo retangular (caso tenha um noque), empurrando a arma para a frente como se estivesse jogando sinuca. Aparentemente ela desenvolveu um ritmo cuidadoso no qual ela abaixa o escudo pelo mínimo instante necessário para dar o ataque fatal. De todo modo, quando a personagem usa esta manobra para fazer um ataque, ela não mais sofre uma penalidade por usar uma arma enquanto se beneficia de seu escudo, e seu escudo adiciona +1 dado de defesa contra o alvo do ataque.

{\bf Desvantagem:} A concentração e coordenação de escudo e arma da personagem deixa ela aberta a ataques pelos flancos e costas, ela não pode defender ataques que não venham de sua frente durante esta ação.
\begin{itemize}[noitemsep]
\item Teste: Destreza + Armas Brancas + Bônus do Escudo
\item Dificuldade: 6
\item Dano: Força
\item Ações: 1
\end{itemize}

\subparagraph{\bf ••• Carga com Escudo:}
A personagem avança, escudo abaixado em sua frente, e colide com a linha inimiga. O ataque inflige dano contusivo, mas se um único sucesso for obtido no ataque, a personagem derruba o inimigo. A personagem faz um teste de Destreza + Esportes, e um oponente que sofre este ataque faz um teste reflexivo de Destreza + Esquiva. A distância máxima percorrida pelo ataque equivale ao número de ações gastas para percorrer essa distância, a critério do mestre. A critério do mestre, essa manobra pode ser usada contra múltiplos oponentes, desde que eles estejam próximos o bastante um do outro. Cada oponente adicional mirado acarreta em -1 na rolagem de ataque, e o dano rolado é distribuído igualmente entre os que foram atingidos. Aqueles que não sofrem dano não são derrubados, mesmo que seus companheiros sejam.
\begin{itemize}[noitemsep]
\item Teste: Destreza + Esportes
\item Dificuldade: 6
\item Dano: Força + Bônus do escudo
\item Ações: Varia com a distância
\end{itemize}

\subparagraph{\bf •••• Defender Rajada:}
A personagem desenvolve seus reflexos com o escudo, aprendendo a bloquear múltiplos golpes em sucessão. Uma personagem defendendo-se (ou defendendo um aliado, caso se aplique) de múltiplos ataques de um oponente com duas armas ou múltiplos oponentes poderá dividir sua parada de dados pela metade, usando a maior, caso não seja uma divisão igual, para a primeira defesa, e a menor para a segunda. O bônus de defesa do escudo é dividido juntamente com a parada de dados (a maior metade indo para a primeira ação, caso não seja uma divisão igual). A personagem não pode usar esta manobra para defender ataques de ações sucessivas, apenas ataques executados durante a mesma ação.
\begin{itemize}[noitemsep]
\item Teste: Destreza + Armas Brancas + Bônus do Escudo
\item Dificuldade: 6
\item Dano: Nenhum
\item Ações: 1
\end{itemize}

\subparagraph{\bf ••••• Mantenha-se Forte:}
A personagem afunda os pés, levanta seu escudo, e permanece como uma parede humana contra uma saraivada de ataques. Se beneficiando de sua parada de dados de Esquiva/Bloqueio completa e seu bônus de defesa do escudo contra ataques feitos de uma única direção (num ângulo de 120º, ou simplesmente frente e esquerda ou direita). Note que enquanto um único inimigo pode ser capaz de mover em torno das defesas da personagem, não mais que três personagens podem atacar a personagem de uma direção que ela não possa defender totalmente com esta manobra.

{\bf Desvantagem:} Usar essa manobra requer um exercício incrível de vontade. Uma personagem deve gastar um ponto de Força de Vontade para ganhar os benefícios desta manobra por um turno. Além disso, nenhuma ação, exceto bloqueio e esquiva, pode ser executada durante este turno, ou o bônus será perdido.
\begin{itemize}[noitemsep]
\item Teste: Destreza + Armas Brancas + Bônus do Escudo
\item Dificuldade: 6
\item Dano: Nenhum
\item Ações: Um turno
\end{itemize}