\titulo{Capoeira}

Quando aprende o primeiro nível de Capoeira, o jogador recebe as manobras "Ginga" e "Seguir o Toque", independente de estilo. Além destas, ele também recebe mais uma manobra à sua escolha (obviamente, desde que possua os pré-requisitos necessários). Para cada novo ponto em Capoeira depois do primeiro, o jogador pode escolher mais duas manobras que seu personagem domina.

Há dois estilos de Capoeira: Angola (Suave) e Regional (Duro), cada um deles está relacionado a um determinado grupo de Habilidades que seus praticantes devem possuir em maior ou menor grau. Para detalhes sobre estas Habilidades auxiliares, e como elas se relacionam com o nível de Capoeira do personagem, veja a descrição de cada nível de Capoeira.

\subsection{Perícia}
• Cordel Cru. Você está dando seus primeiros passos na arte ancestral da capoeiragem e não teve ainda mais do que um vislumbre de seu potencial. Antes de poder comprar este nível de Capoeira, o personagem já deve possuir pelo menos: Performance 1 ou Arte 1; e também Esportes 1 ou Acrobacia 1.

•• Batizado. Você acaba de ser reconhecido como um verdadeiro Capoeira, muito provavelmente passou pelo ritual do batizado e recebeu um nome pelo qual é reconhecido nas Rodas. Antes de poder comprar este nível de Capoeira, o personagem deve possuir pelo menos: Performance ou Arte 1; e também Esportes ou Acrobacia 1.

••• Graduado. Você já é um capoeirista com alguma experiência, provavelmente costuma compor a bateria de Rodas de Capoeira reconhecidas e talvez já esteja conduzindo suas próprias Rodas e já esteja dando algumas dicas para capoeiristas menos experientes. Antes de poder comprar este nível de Capoeira, o personagem deve possuir pelo menos: Performance ou Arte 2; e também Esportes ou Acrobacia 2.


•••• Contramestre. Você certamente conquistou reconhecimento entre os adeptos da Capoeira. Costuma conduzir suas próprias Rodas e batizados. Antes de poder comprar este nível o personagem deve possuir pelo menos: Performance ou Arte 2; e também Esportes ou Acrobacia 2.


••••• Mestre. Você é uma lenda viva entre os praticantes de Capoeira de todo o mundo e se orgulha da maestria com a qual constrói e toca seus próprios instrumentos. Capoeiristas experientes amontoam-se em torno das Rodas que você conduz para poderem aprender com suas dicas e histórias. Você cria suas próprias canções que são espalhadas por Rodas de Capoeira em volta do planeta pelos Contra-Mestres e Graduados que você formou pessoalmente. Antes de poder comprar este nível de Capoeira, o personagem deve possuir pelo menos: Performance ou Arte 3; e também Esportes ou Acrobacia 3; e, por fim, também Ofícios 1.

Possuído por: Ativistas do movimento negro, adeptos das religiões de matriz africana, quilombolas, afrodescendentes, malandros, artistas marciais brasileiros, moradores da periferia das grandes metrópoles latino-americanas, entusiastas da cultura brasileira pelo mundo.

Especialidades: Manobras Evasivas, Pernada Devastadora, Velhacaria (manobras baseadas na esperteza e malandragem, para confundir e enganar o oponente), Maculelê (uso de armas brancas na Capoeira), Exibições, Desarmar.

\subparagraph{\bf Ginga:} 
Este é o movimento mais básico da Capoeira. Como se estivesse empenhado em uma dança, um capoeirista jamais permanece parado na frente de um oponente: o objetivo da ginga é distrair e confundir o adversário com fintas e blefes, enquanto se busca uma brecha em sua defesa. No início de cada turno, logo antes da rolagem de Iniciativa simplesmente compare a atual soma de seu Raciocínio + Capoeira com a atual soma de Raciocínio + Combate do alvo: se a sua soma for maior do que a dele, você terá um bônus de +1 na sua base de iniciativa durante este turno. Observe que caso ele também use Capoeira, se a soma dele for maior ele terá esta vantagem. No caso de somas iguais, ninguém ganha nenhum bônus (em um empate tanto faz se um oponente é ou não Capoeirista). Mínimo de Habilidade: Capoeira 1 (qualquer estilo); todo Capoeirista recebe esta manobra gratuitamente. Ações: Reflexiva.

\subparagraph{\bf Seguir o Toque:} 
Enquanto está numa roda de capoeira, o capoeirista sente o toque do berimbau e o canto dos companheiros na Roda, e flui em conjunto com eles. Desde que esteja em uma verdadeira Roda de Capoeira, onde os participantes acompanham, batendo palmas, o ritmo marcado por uma bateria formada por pelo menos um berimbau e um atabaque e fazem ecoar o canto daquele que conduz a roda, o capoeirista encontra-se em um estado de transe que o induz à completa concentração. Capoeiristas de inclinações mais místicas costumam relatar que é como se pudesse compartilhar da experiência de capoeiristas do passado que de algum lugar observam o andamento da Roda. Em termos de sistema, sob essas condições especificas, o praticante de capoeira recebe +1 dado em todos os seus testes de Capoeira, Esportes, e Acrobacia. Mínimo de Habilidade: Capoeira 1 (qualquer estilo); todo Capoeirista recebe esta manobra gratuitamente. Ações: Reflexiva.

\subparagraph{\bf Chute Básico:}
(Benção, Meia Lua de Frente, Chapa, Esporão) Golpes com as pernas compõem a maior parte do arsenal de qualquer capoeirista, concentrando sua energia em um chute eficaz, o artista marcial pode desferir golpes vigorosos e quase sempre certeiros. Habilidade Mínima: Capoeira 1 (qualquer estilo). Teste: Destreza + Capoeira. Dificuldade: 5. Dano: Força + 1 (Contusão). Ações: 1.

\subparagraph{\bf Contragolpe:}
(Vingativa, Arrastão, Tesoura de Frente) Evitando o golpe, o Capoeirista redireciona a força do atacante contra ele. Em termos de jogo, ao invés de fazer uma Defesa ou Esquiva, o Capoeirista rola Destreza + Capoeira: cada sucesso subtrai um sucesso do teste do atacante. Se o personagem obtiver mais sucessos do que o atacante, então o atacante deve rolar sua Destreza (dificuldade 8) ou cairá no chão (ou contra uma superfície próxima), levando um dano equivalente a sua própria Força. Mínimo de Habilidade: Capoeira (qualquer estilo) 2. Teste: Destreza + Capoeira. Dificuldade: 6. Dano: Força do Atacante (Contusão). Ações: 1.

\subparagraph{\bf Manobra Evasiva:}
(Aú, Macaco, Malandro, Palhaço, Negativa) Desviando-se dos golpes com graça ou agilidade o capoeirista livra-se de dano. Naturalmente, ele deve ser capaz de perceber os ataques, em primeiro lugar. Esta manobra atua como um teste de Esquiva, com três dados de bônus e testando Destreza + Capoeira ao invés de Destreza + Esportes ou Destreza + Acrobacia. Habilidade Mínima: Capoeira 1 (qualquer estilo). Teste: Destreza + Capoeira + 3 dados. Dificuldade: 6 para Regional / 5 para Angola. Ações: 1.

\subparagraph{\bf Rasteira:}
Um golpe rápido e preciso, que se bem-sucedido leva o oponente ao chão, obrigando-o a gastar uma ação para se levantar. Para a rasteira levar alguém ao chão basta que o Capoeirista consiga um ou mais sucessos além dos sucessos de Defesa ou Esquiva do oponente. O dano deste golpe é baseado na força do próprio oponente, e só é aplicado no exato momento do impacto dele com o chão (se ele não cair, não há rolagem de danos); fora isto, as regras normais de dano se aplicam. Mínimo de Habilidade: Capoeira (Angola) 1 / Capoeira (Regional) 2. Dificuldade: 8. Dano: Força do Oponente (Contusão). Ações: 1

\subparagraph{\bf Gingado Feral:}
Manobra exclusiva para Metamorfos. Garou e outros Bête mais experientes do que meros iniciantes, podem aprender a gingar e fluir perfeitamente em sua forma "quase-humana". Um Metamorfo que aprenda esta Técnica ignora a penalidade usar Capoeira em sua forma "quase-humana" (Glabro ou equivalente). Continua sendo impossível usar manobras de Capoeira em quaisquer outras formas. Habilidade Mínima: Capoeira 2 (qualquer estilo). Testes não são necessários, uma vez aprendida esta capacidade é permanente e está sempre ativa.

\subparagraph{\bf Golpe Letal:}
(Ponteira, Martelo, Godeme, Forquilha, Joelhada ou Cotovelada) Visando um órgão, articulação ou outro local incapacitante, um ataque preciso causa um efeito devastador no alvo. Habilidade Mínima: Capoeira (qualquer estilo) 3. Teste: Destreza + Capoeira. Dificuldade: 5 (Regional)/ 6 (Angola). Dano: Força + 2 (Letal). Ações: 1

\subparagraph{\bf Recompor:}
(Martelo de Chão, “S” Dobrado, Meia-Lua em Queda de Rins) Um capoeirista pode ser perigoso mesmo depois de ser levado ao chão, esta manobra permite ao praticante de capoeira se reerguer já desferindo um golpe contra seu oponente, sem a necessidade de perder uma ação apenas para se levantar após ser lançado ao chão ou ser alvo de uma rasteira. 
	Habilidade Mínima: Capoeira 3 (qualquer estilo)
    Teste: Destreza + Artes Marciais
    Dificuldade: 7 (Regional) / 6 (Angola)
    Dano: Força + 1 (Contusão). 
    Ações: 1.

\subparagraph{\bf Golpear Atado:}
A Capoeira é uma técnica marcial desenvolvida por descendentes africanos feitos cativos no Brasil, ou seja, foi pensada para ser efetiva em situações de franca desvantagem. Alguns capoeiristas especialmente competentes resgatam essa tradição praticando o suficiente para manterem sua efetividade plena em combate mesmo com as mãos ou os pés atados. Com essa manobra é possível ignorar qualquer penalidade por estar com as mãos ou os pés atados, desde que não estejam ambos atados ao mesmo tempo, sejam por cordas, correntes, algemas ou grilhões. 
	Habilidade Mínima: Capoeira 4 (qualquer estilo), Esportes ou Acrobacia 3. 
    Testes não são necessários, uma vez aprendidos esta capacidade é permanente e está sempre ativa.

\subparagraph{\bf Maculelê:}
Uma técnica tradicional e letal que funde a habilidade com o uso de Armas Brancas com manobras de Capoeira. Ao adquirir essa manobra o praticante elege uma arma branca específica: a escolha tradicional é o facão, mas o porrete, a faca, a navalha e os oxés (pequenos machados de lâminas gêmeas associados a Xangô) são alternativas razoáveis. Sempre que estiver empunhando duas armas idênticas do tipo especificado o Capoeirista recebe um bônus de -1 na dificuldade dos testes para atacar ou bloquear usando estas armas (veja o quadro “Lutando com Duas Armas Brancas”). Importante: se o Capoeirista estiver usando apenas uma arma (mesmo que ela seja do tipo escolhido) ele não recebe este bônus. Ele também não recebe o bônus caso esteja usando um tipo diferente de arma em cada mão; a técnica exige o uso de duas armas do tipo escolhido, uma em cada mão. 
	Habilidade Mínima: Capoeira (Regional) 3/ Capoeira (Angola) 4, Armas Brancas 3
    Teste: Destreza + Armas Branca
    Dificuldade: -1.

\subparagraph{\bf Música Interior:}
O capoeirista não precisa mais estar em uma roda de capoeira para induzir o transe que experimenta durante a Manobra "Seguir o Toque". Ele carrega a música e a vibração da Roda consigo sempre que precisar dela. 
	Habilidade Mínima: Capoeira 4 (qualquer estilo), Performance ou Artes 3 e Ofícios 1. 
    Ações: Reflexiva.

\subparagraph{\bf Truque de Mestre:}
Alguns Mestres realmente habilidosos são capazes de dominar até mesmo manobras exclusivas que não se esperaria estarem ao seu alcance. Se um personagem definir “Truque de Mestre” como uma de suas manobras ao chegar ao nível 5 de Capoeira, ele deve imediatamente escolher uma manobra da lista de “manobras exclusivas” do estilo que não é o dele (em outras palavras, um Mestre de Regional escolhe uma manobra exclusiva de Angola, e um Mestre de Angola escolhe uma manobra exclusiva de Regional). Ele aprende esta manobra e poderá usá-la normalmente daqui por diante. É possível escolher “Truque de Mestre” duas vezes: um personagem que faça isso simplesmente aprenderá um total de duas manobras exclusivas “do outro estilo” (e obviamente, não poderá escolher mais nenhuma manobra, pois já terá as duas a quem tem direito por chegar ao nível 5). 
	Habilidade Mínima: Capoeira 5 (qualquer tipo).

\subparagraph{\bf Mestre Consagrado:}
Aqueles Capoeiristas que se esforçaram para compreender tanto a Regional quanto a Angola ao longo de toda a sua vivência na Capoeira, podem romper as fronteiras entre os diferentes estilos, finalmente colhendo os frutos de sua dedicação ao atingirem o nível de Mestre. A partir deste ponto, toda vez que o Mestre usar manobras comuns aos dois estilos, ele fará suas rolagens com a dificuldade mais baixa entre as listadas, não importando se ele adquiriu originalmente o estilo Angola ou Regional. Uma vez aprendida, esta habilidade está sempre ativa. 
	Habilidade Mínima: Capoeira 5 (qualquer tipo).

\subparagraph{\bf Jogo de Santa Maria:}
Esta é uma técnica lendária, que somente alguns poucos Mestres conseguiram dominar e que na atualidade é considerada perdida. O capoeirista é capaz de lutar com navalhas em suas mãos e entre o dedão do pé e o segundo dedo. A precisão é tamanha, que permite usar também apenas a lâmina de uma navalha entre os dedos, e sem cortar a si mesmo. Se estiver com pelo menos uma navalha em quaisquer mãos e outra em pelo menos um dos pés, o Capoeira pode escolher tornar o dano de quaisquer manobra que realize em dano Letal. Adicionalmente, devido a essas lâminas serem extremamente cortantes, todas as rolagens do alvo para absorver o dano desferido dessa forma recebem +1 na dificuldade de seus testes. A única desvantagem é que o mínimo descontrole pode ser fatal: caso sofra uma Falha Crítica em qualquer Técnica em que esteja usando uma lâmina desse modo, o Capoeirista irá ferir a si mesmo (perdendo 1 Nível de Vitalidade Letal automaticamente, sem direito a teste de absorção de dano), além de perder a lâmina (que provavelmente caiu em algum lugar fora de sua visão e de seu alcance, impedindo que ele prossiga utilizando essa técnica). 
	Habilidade Mínima: Capoeira 5 (qualquer tipo).

\subsection{Manobras Exclusivas de Capoeira Regional}

\subparagraph{\bf Chute Giratório:}
(Armada, Meia Lua de Compasso, Rabo de Arraia) Com um giro completo do corpo o praticante desfere um ataque especialmente poderoso. 
	Habilidade Mínima: Capoeira (Regional) 2
    Teste: Destreza + Capoeira
    Dificuldade: 6
    Dano: Força + 3 (Contusão)
    Ações: 1.

\subparagraph{\bf Pernada Devastadora:}
(Folha Seca, Vôo do Morcego, Parafuso) Um chute voador devastador. Arremessando-se através do ar, o artista marcial focaliza sua massa em um golpe potente bastante para terminar a maioria de lutas imediatamente. Um personagem não pode usar essa manobra mais do que uma vez a cada cinco turnos, pois ela exige foco e total compromisso com o golpe. Use as regras normais de sucessos extras adicionarem dados de dano, com uma pequena modificação: cada sucesso extra adiciona dois dados ao dano do golpe, ao invés de apenas um.
	Habilidade Mínima: Capoeira (Regional) 3
    Teste: Destreza + Capoeira
    Dificuldade: 7
    Dano: Força +3 +Dados Extras para Dano (veja a descrição).(Contusão).
    Ações: 1.

\subsection{Manobras Exclusivas de Capoeira de Angola}

\subparagraph{\bf Cabeçada:}
Em determinadas situações, o capoeirista pode desferir inesperadamente uma cabeçada no rosto ou estômago de um adversário. Obviamente, este movimento funciona apenas quando ambos estão bem próximos, como em uma tentativa de agarrar ou imobilizar o capoeirista. A menos que o atacante tenha uma cabeça com chifres, capacete cravado, crânio de aço, ou outro instrumento letal revestindo seu crânio, esta manobra inflige apenas dano por Contusão. O principal benefício do ataque vem da surpresa: qualquer tentativa de Esquiva ou Bloqueio contra esta manobra deve ser realizada com +1 na dificuldade. Infelizmente, se o alvo absorver todo o dano da cabeçada o atacante ficará atordoado por um turno. Usar uma cabeçada contra uma parte inflexível de um oponente é uma péssima ideia — seja uma pessoa de capacete, alguém vestido com um elmo ou armadura de placas metálicas medieval ou romana, ou em casos mais inusitados, “coisas” estranhas como um cyborg ou personagens com exoesqueletos metálicos ou quitinosos. Qualquer pessoa estúpida o suficiente para desferir uma cabeçada contra um oponente ou superfície inflexível sofre o dano que teria infligido e também fica atordoado por um turno. 
	Habilidade Mínima: Capoeira 1 (Angola). 
	Teste: Destreza + Capoeira
    Dificuldade: 6
    Dano: Força + 1 (Contusão)
    Ações: 1

\subparagraph{\bf Bloqueio de Juntas:}
(Quebra-joelho, Crucifixo, Arpão) Um ataque habilidoso contra as articulações do oponente para debilitá-lo, permitindo deslocar ou quebrar um de seus membros ou articulações. O jogador testa Destreza + Capoeira para atacar seu oponente, se for bem-sucedido, então pode imediatamente (sem precisar dividir sua parada de dados) testar Destreza + Capoeira para infligir dano. Cada sucesso neste teste inflige um Nível de dano Letal, além de levar o alvo ao chão. 
	Habilidade Mínima: Capoeira (Angola) 3. 
	Teste: Destreza + Capoeira
    Dificuldade: 7
    Dano: ver descrição
    Ações: 1

\subparagraph{\bf Desarme:}
Para retirar uma arma das mãos de um inimigo e fazê-la cair no chão, o Capoeirista deve conseguir pelo menos três sucessos (depois de qualquer rolagem de Esquiva ou Defesa feita pelo oponente); com cinco sucessos, ele pode capturar a arma do oponente para si e usá-la no turno seguinte.
	Habilidade Mínima: Capoeira (Angola) 3
    Teste: Destreza + Capoeira
    Dificuldade: 7
    Dano: Força (Contusão)
    Ações: 1