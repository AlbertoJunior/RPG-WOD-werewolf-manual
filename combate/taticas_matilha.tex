\titulo{Táticas de Matilha}

\begin{multicols}{2}
\subsection{\bf Combate}

\subparagraph{\bf Arrancar Pelos}
Geralmente executada por dois Garou em rápida sucessão, este consiste em o primeiro membro da matilha arrancar a proteção natural (ou mesmo a armadura artificial) de um oponente. Deixando um ponto vulnerável para o membro seguinte atacar. 
O primeiro atacante faz um teste de Destreza + Briga para um ataque com as garras, arruinando a pelagem ou a proteção de um oponente e, esperamos, levando um bom naco dela consigo (a dificuldade do teste é igual a 7) Para cada dois sucessos obtidos no dano (antes da absorção), O alvo perderá um dado dos testes de absorção aquela área. O membro seguinte da matilha poderá então atacar normalmente, mas sua dificuldade aumentará em dois pontos, pois ele terá como alvo a mesma área que o primeiro atacante atingiu. Essa penalidade persistirá até o alvo poder reparar o dano ou arranjar nova armadura.
\begin{itemize}[noitemsep]
\item Companheiros necessários: 2 
\item Executável sozinho? Sim
\end{itemize}

\subparagraph{\bf Ataque Feroz}
Também chamado de “pilha de cachorro” pelos Roedores, este ataque brutal requer pelo menos três membros da matilha, embora possa envolver cinco ou mais no caso de oponente para derrubá-lo no chão, depois do que o resto da matilha avança sobre ele na forma Hispo ou Lupina afim de mordê-lo enquanto estiver caído.

\bf{Sistema:} O primeiro Garou executa um ataque normal. Como encontrão ou rasteira. Depois, seus companheiros de matilha cercam o adversário caído e mordem qualquer parte que estiver ao alcance. Este ataque pode matar a maioria dos inimigos em segundos, mas os oponentes que não forem mortos imediatamente deverão fazer teste de Força + Esporte (dificuldade 4 + 1 para cada Garou envolvido, Máximo de 10) para se levantar. 
\begin{itemize}[noitemsep]
\item Companheiros necessários: 2 
\item Executável sozinho? Sim
\end{itemize}

\subparagraph{\bf Ataque Surpresa:}
Uma variação do “Um por Um”, o Ataque Surpresa permite a vários lobisomens atacarem um único alvo sem que um membro se coloque em risco direto. A maioria dos agressores atravessa a Penumbra e se movem para trás do alvo. Após aguardar um momento antes combinado, o único lobisomem restante no mundo material corre até o alvo, faz um ataque e se distancia novamente. Assim que o lobisomem fizer seu ataque, seus companheiros de matilha saem da Umbra e lançam seus ataques ao alvo. Enquanto o oponente se volta para enfrentar esses novos agressores, o primeiro lobisomem volta a dar outro ataque, agora pelas costas. Devido à natureza enganadora de caminhar através da Película rapidamente, essa tática só é útil se todos os envolvidos forem veteranos em entrar e sair do mundo espiritual.

{\bf Sistema:} O lobisomem no mundo material deve gastar pelo menos dois pontos de Fúria para correr até o alvo, atacar e fugir novamente. Enquanto ele faz seu ataque, seus companheiros de matilha simplesmente percorrem atalhos novamente e atacam os lados desprotegidos do alvo com um bônus normal de -2 nas dificuldades.
\begin{itemize}[noitemsep]
\item Companheiros necessários: 2 ou mais
\item Executável sozinho? Não
\end{itemize}

\subparagraph{\bf Cerco}
Os Lobos não atacam a presa imediatamente, mesmo estando em alcateias. Em vez disso, eles perseguem e importunam sua futura refeição até ela desfalecer devido à exaustão. Os lobisomens usam um método similar para atacar e confundir os inimigos.
O cerco requer pelo menos quatro Garou: um à frente da, um atrás e de cada lado. O Garou mais na retaguarda persegue a presa e a conduz para um dos Garou laterais ou para o líder. O assediador na retaguarda então reassume sua posição e o novo assediador surpreende a presa, rosnando e mordiscando e persegue-a   em direção a um outros Garou, e assim por diante até que a vítima fique exausta. Uma vítima humana perderá um ponto de Força de Vontade toda vez que for “passada” para outro lobisomem.

\bf{Sistema:} Faça um teste de Destreza + Esporte (dificuldade 5) tanto para o Garou em perseguição quanto a presa.  Se obtiver mais sucesso, o Garou perseguirá a vítima com sucesso e vai passá-la para outro Garou. Em geral, o Garou precisa vencer cinco desses testes para forçar a vítima a se aproximar o suficiente de outro lobisomem e passá-la adiante.
Entretanto se obtiver mais sucesso a presa abrirá distância em relação a seu atacante e poderá escapar. Faça os mesmos testes novamente, mas desta vez o jogador que interpreta o Garou acrescentará os sucessos excedentes da presa a sua dificuldade. Por exemplo, se o jogador que interpreta o Garou acrescentará quatro a sua dificuldade, que passará a ser igual a 9. O Garou deve exceder os sucessos da presa – contra a dificuldade ajustada – para alcançá-la e começar a desviá-la em direção aos caçadores na espera. Se a presa escapar o Garou deverá recorrer ao rastreamento para encontrá-la.
Se a presa optar por resistir e lutar, terá início um combate normal.  Ao contrário dos lobos normais, os lobisomens não recuarão diante da presa. É a natureza deles lutar, não fugir.
O cerco pode ser executado em qualquer forma quadrúpede Algumas tribos, principalmente a Cria de Fenris e os Senhores da Sombras, usam este método para sequestrar filhotes de suas famílias humanas ou lupinas. Eles cansam o filhote até a exaustão o que o torna muito mais fácil de se capturar e transportar para seu novo lar.
\begin{itemize}[noitemsep]
\item Companheiros necessários: 4 
\item Executável sozinho? Não
\end{itemize}

\subparagraph{\bf Osso da Sorte}
Um ou mais Garou agarram as extremidades de um oponente e o parte ao meio. Não é muito sutil, é verdade, mas é uma maneira espetacular de obrigar um adversário não cooperativo a falar ou de fazer os inimigos remanescentes tremerem de medo. 
Cada jogador deve fazer um teste de Destreza + Briga para seu personagem agarrar uma extremidade. A dificuldade inicial é igual a 6 e será reduzida em um ponto a cada Garou excedente (pois a capacidade do oponente de se se esquivar vai diminuindo). Depois de imobilizadas toas as extremidades possíveis, cada jogador fará um teste de Força para avaliar o dano. Este dano é considerado letal. Os Garou envolvidos podem escolher entre puxar devagar como um método de tortura (caso em que dano é considerado contundente) ou simplesmente dar um puxão rápido, maximizando o dano. Se qualquer um dos Garou infligir mais do que três níveis de dano depois de absorção, a extremidade será fraturada ou decepada (a critério do Narrador). 
É possível executar está manobra nas formas Lupina ou Hispo; só requer que a vítima seja derrubada primeira. 
\begin{itemize}[noitemsep]
\item Companheiros necessários: 2 
\item Executável sozinho? Não
\end{itemize}

\subparagraph{\bf Passe à Frente:}
O Passe à Frente tira vantagem da discrepância de tamanho e massa entre as diferentes formas para permitir que um lobisomem literalmente arremesse outro a uma distância significativa. Um lobisomem assume a forma Crinos, segura e arremessa outro membro da matilha, que normalmente está na forma Hominídea ou Lupina. Essa tática normalmente é usada quando uma massa de inimigo está bloqueando o caminho da matilha, quando eles precisam urgentemente chegar até algo ou alguém atrás deles e foi usada primeiro por um grupo de Fianna ingleses com alguma experiência de rugby antes de sua Primeira Mudança. Se o lobisomem arremessado estiver esperando lutar contra o que quer que seja que ele tenha sido arremessado, ele normalmente muda sua forma para Crinos segundos antes do impacto, esperando que o momento drasticamente ampliado derrube o oponente.

{\bf Sistema:} O arremessador testa Força + Esportes (dificuldade 3 + Vigor do Garou arremessado). Caso o lobisomem arremessado esteja tentando derrubar algo ou alguém, ele testa sua Destreza + Esporte para acertar o alvo, que pode esquivar, e causa Vigor + sucessos obtidos pelo arremessador de dano contusivo no alvo. Do contrário, o jogador testa Destreza + Esporte (dificuldade 6) para permitir que seu personagem caia em pé.
\begin{itemize}[noitemsep]
\item Companheiros necessários: 2 
\item Executável sozinho? Não
\end{itemize}

\subparagraph{\bf Um por Um:}
Algumas vezes dividir para enfrentar vários inimigos não é a maneira mais efetiva de destruí-los. Ao invés disso, a matilha se concentra em um membro por vez, esperando derrubá-los um a cada vez. Normalmente um membro da matilha engaja o inimigo escolhido diretamente, lançando ataques múltiplos, governados pela Fúria, enquanto seus companheiros fazem seu melhor para rasgar o oponente pelas costas e flancos. Uma vez que o inimigo cai, eles se movem para fazer o mesmo no próximo oponente. 

{\bf Sistema:} O agressor principal ataca o oponente normalmente, a não ser pelo fato de que ele deve gastar Fúria o suficiente para ganhar um número de ataques igual ao número de atacantes que tentam derrubar seu oponente. Além disso, ele também não se pode defender em nenhuma de suas ações, todas devem ser ataques diretos contra o adversário. Uma vez que todas essas condições sejam preenchidas, os outros lobisomens podem atacar com -2 de dificuldade. 
\begin{itemize}[noitemsep]
\item Companheiros necessários: 3 ou mais
\item Executável sozinho? Não
\end{itemize}

\subsection{Sociais}

\subparagraph{\bf Falso Salvador:}
Muitas matilhas acham o uso dessa tática uma forma efetiva de se aproximar de uma possível fonte de informação e ajuda que eles não acham que responderia positivamente a uma abordagem direta. Dois ou mais membros da matilha planejam pegar a vítima em um lugar ermo e ameaça ou realmente o ataca. Normalmente os lobisomens maiores e mais agressivos são escolhidos para essa tarefa. Então outro membro da matilha “está por ali” e consegue “afugentar” os agressores. Esse falso salvador recebe um bônus de -1 em todos os testes sociais com a vítima a partir de então, a menos que seja visto com seus companheiros. Nesse caso, o bônus muda para uma penalidade de +2 nas dificuldades.
\begin{itemize}[noitemsep]
\item Companheiros necessários: 3 ou mais
\item Executável sozinho? Não
\end{itemize}

\subparagraph{\bf Magnetismo Animal:}
Quando um lobisomem caça sozinho por um parceiro, ele pode usar Atração Animal (veja Lobisomem, pág. 199) para encontrar um parceiro de cama. Quando uma matilha inteira sai à procura, é melhor que a cidade cuide de seus filhos e filhas. O efeito cumulativo de um grupo de lobisomens todos evidentemente checando os talentos locais pode acabar aumentando o nível de tensão sexual em um bar ou boate a tal ponto que se torna mais fácil para todo mundo seduzir qualquer pessoa. Os lobisomens, claro, estão na melhor situação. Todo lobisomem do sexo apropriado para o parceiro que o lobisomem está tentando seduzir e que esteja participando ativamente do flerte e da sedução dos humanos à sua volta diminui a dificuldade do teste de Atração Animal em um. 
\begin{itemize}[noitemsep]
\item Companheiros necessários: 2 ou mais
\item Executável sozinho? Sim
\end{itemize}

\subparagraph{\bf Policial Mau, Policial Pior:}
Muitas matilhas perceberam que fazer o jogo de policial bom e policial ruim pode ser bastante efetivo se um dos participantes for um lobisomem com Fúria alta o suficiente para os humanos à sua volta se sentirem desconfortáveis. Usando tal lobisomem como um “policial pior” dá ao “policial mau” um dado extra nos testes de Manipulação que ele fizer para persuadir o sujeito do interrogatório a falar. 
\begin{itemize}[noitemsep]
\item Companheiros necessários: 2
\item Executável sozinho? Não
\end{itemize}

\end{multicols}