\titulo{Combate Próximo}

\subparagraph*{\bf Descrição:} É qualquer luta em que os combatentes usam as mãos nuas (ou garras, ou dentes) ou uma arma d mão (facas, garrafas quebradas, machados de prata, etc). Os combatentes têm de estar próximos o bastante para usar seus corpos como armas (obviamente). Algumas armas brancas, como as armas de haste, melhoram o alcance do personagem.

\subparagraph{\bf Ataque Pelos Flancos:}
Atacar os flancos de um oponente reduz a dificuldade em um ponto. Atacar pela retaguarda reduz a dificuldade do atacante em dois pontos.

\subparagraph{\bf Chute:}
Os ataques com chutes variam em estilo desde um pontapé rápido na virilha até elaborados giros no ar. O Narrador deve se sentir à vontade para ajustar o dano e a dificuldade a fim de compensar a complexidade da manobra. O sistema aqui apresentado tem como pressuposto um ataque com chute direto.
O dano causado por um chute é geralmente considerado contundente, mas o Garou na forma Crinos que chutar um ser humano normal pode causar dano letal.
{\bf Empregável por:} Hominídea / Crinos.
\begin{itemize}[noitemsep]
\item Teste: Destreza + Briga
\item Dificuldade: 7
\item Dano: Força + 1
\item Ações: 1
\end{itemize}

\subparagraph{\bf Comprimento de Arma:}
O personagem armado com uma faca que enfrenta outro armado com uma espada está em franca desvantagem simplesmente porque precisa entrar na zona de alcance de seu oponente a fim de causar-lhe dano. Ao enfrentar um oponente armado com uma arma de comprimento apreciável, o personagem com a arma mais curta perderá um dado de sua parada de ataque a fim de refletir a aproximação necessária para atingir o oponente.

\subparagraph{\bf Desarme:}
Semelhante a aparar, o personagem tenta usar uma arma para arrancar a arma de seu oponente. O atacante faz o teste de ataque como sempre, mas submetido a uma penalidade igual a +1 sobre a dificuldade. Se a quantidade de sucessos do atacante for igual ou superior à pontuação de Força de seu oponente, este deixará a arma cair. Se não conseguir sucessos suficientes, o atacante ainda infligirá dano normalmente. Uma falha critica nesse teste geralmente significa que o atacante deixou cair sua arma.
É possível, apesar de muito mais difícil, executar esta manobra sem uma arma. Nesse caso, o teste é Destreza + Briga, a dificuldade é igual a 9 e o personagem deve remover um dado de sua parada de ataque como se estivesse se deslocando para dentro da área de alcance de um oponente com uma arma mais longa.
{\bf Empregável por:} Hominídea / Crinos
\begin{itemize}[noitemsep]
\item Teste: Destreza + Armas Brancas
\item Dificuldade: + 1
\item Dano: Especial
\item Ações: 1
\end{itemize}

\subparagraph{\bf Encontrão:}\label{encontrao}
Esta perigosa manobra pode causar mais dano ao atacante que a seu oponente. O atacante corre a toda velocidade em direção ao oponente na esperança de adquirir impulso suficiente para derrubar o dito oponente. Os dois combatentes precisam ter sucesso nos testes de Destreza + Esportes (Dificuldade 6 para o atacante, 6 + os sucessos do atacante para o alvo) para não serem derrubados.
No caso de uma falha critica, duas coisas podem acontecer. O atacante tropeça e cai ou choca-se impetuosamente contra o alvo e é jogado para trás, o que deixa o alvo intacto mas provoca no atacante o Vigor do alvo em dados de dano.
Todo dano provocado por este ataque é considerando contundente. Se o ataque for empregado por um Garou nas formas Crinos ou Hispo contra um ser humano desprotegido, o dano poderá ser considerado letal (a critério do Narrador).
{\bf Empregável por:} Qualquer forma.
\begin{itemize}[noitemsep]
\item Teste: Destreza + Briga
\item Dificuldade: 7
\item Dano: Força
\item Ações: 1
\end{itemize}

\subparagraph{\bf Engalfinhamento:}
Engalfinhar-se é o ato de se atracar com um oponente e segurá-lo com a intenção de imobilizá-lo ou machucá-lo. Um ataque desse tipo, com a intenção de imobilizar é chamado de imobilização.
Qualquer um dos ataques começa com o atacante passando num teste de Força + Briga. O sucesso indica que o atacante engalfinhou-se com o oponente. No caso de uma chave, o atacante poderá infligir dano igual a sua Força, a partir do turno seguinte ao engalfinhamento. No caso de uma imobilização, o alvo é seguro até sua próxima ação.
Quando é alvo de um engalfinhamento, o personagem tem duas opções. A primeira é escapar, o que exige um teste resistido de Força + Briga. Como opção do Narrador, os defensores podem fazer testes de Destreza ao invés de Força para escapar à chave ou à imobilização. Se o atacante vencer, o engalfinhamento continuará. Se o defensor vencer, ele se libertará. A outra opção é reverter o golpe. O teste é idêntico ao anterior, mas, para reverter o golpe com sucesso, o defensor terá de conseguir pelo menos dois sucessos a mais que o atacante.
{\bf Empregável por:} Hominídea / Crinos
\begin{itemize}[noitemsep]
\item Teste: Força + Briga
\item Dificuldade: 6
\item Dano: Força ou Nenhum
\item Ações: 1
\end{itemize}

\subparagraph{\bf Garras:}
Outro ataque simples e comumente empregado, este golpe consiste meramente no Garou dilacerar o oponente com as garras. Este ataque causa ferimentos agravados nas formas Crinos e Hispo. Apesar de serem um tanto longas e afiadas nas formas Glabro e Lupina, as unhas de um lobisomem ainda são fracas demais para infligir dano real.

{\bf Empregável por:} Crinos / Hispo.
\begin{itemize}[noitemsep]
\item Teste: Destreza + Brigas
\item Dificuldade: 6
\item Dano: Força + 1
\item Ações: 1
\end{itemize}

\subparagraph{\bf Mordida:}\label{mordida}
Provavelmente a forma de ataque mais básica de todas. O Garou simplesmente enterra as presas na vitima. O dano adicional devido ao posicionamento da mordida (jugular, tecido sensível, etc.) fica a cargo do Narrador. Normalmente, as mordidas provocam ferimentos agravados.
{\bf Empregável por:} Todas exceto Hominídeo (opcionalmente, um lobisomem em Glabro poderia morder a uma dificuldade igual a 8 e causar [Força - 1] pontos de dano)
\begin{itemize}[noitemsep]
\item Teste: Destreza + Briga
\item Dano: Força + 1
\item Dificuldade: 5
\item Ações: 1
\end{itemize}

\subparagraph{\bf Rasteira:}
O personagem usa suas pernas ou uma arma para derrubar o oponente com uma rasteira. É claro que apenas certas armas podem ser usadas desse modo. Como os braços dos lobisomens na forma Crinos são desproporcionalmente longos, eles podem dar uma rasteira num oponente menor usando os braços em vez das pernas. De modo similar, um Garou pode tentar fazer um adversário tropeçar enquanto estiver na forma Hispo ou Lupina, embora isso eleve a dificuldade em um ponto.
A rasteira não provoca dano mas deixará o oponente prostrado se for bem-sucedida.
{\bf Empregável por:} Todas as formas.
\begin{itemize}[noitemsep]
\item Teste: Destreza + Briga
\item Dificuldade: 8
\item Dano: Nenhum
\item Ações: 1
\end{itemize}

\subparagraph{\bf Soco:}
Assim como os chutes, os socos podem ser tão simples quanto um sopapo na cara ou tão impressionantes quanto um murro com os dois punhos, capaz de derrubar um oponente. O Narrador pode decidir acrescentar dados de dano e/ou modificar a dificuldade do atacante no caso de ataque com socos especiais, como os diretos, os curtos ou os ganhos.
O dano de um soco costuma ser contundente, porém, mais uma vez, o Garou na forma Crinos que esmurrar um ser humano poderá provocar dano letal. É a força dos lobisomens!
{\bf Empregável por:} Hominídea / Crinos.
\begin{itemize}[noitemsep]
\item Teste: Destreza + Briga
\item Dificuldade: 6
\item Dano: Força
\item Ações: 1
\end{itemize}